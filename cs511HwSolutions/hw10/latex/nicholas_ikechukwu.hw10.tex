\documentclass{article}
\usepackage[letterpaper]{geometry} % Set the paper size to US Letter
\usepackage{amsmath, amssymb, bussproofs, qtree, booktabs, array, lastpage, fancyhdr}
\usepackage{mdframed}
\usepackage{tcolorbox}
\usepackage{hyperref}
\usepackage{enumitem}



% \defeqine a custom-proof environment

\newenvironment{proof}
{\begin{mdframed}[linewidth=0.5pt]\begin{enumerate}[label=\arabic*.,leftmargin=*]}
{\end{enumerate}\end{mdframed}}

\hypersetup{
  colorlinks=true,
  linkcolor=blue,
  filecolor=magenta,      
  urlcolor=cyan,
  pdftitle={Your Title Here},
  pdfpagemode=FullScreen,
}
 

% Add the new command here
\newcommand{\defeq}{\stackrel{\text{def}}{=}}
\newcommand{\proves}{\vdash}


\pagestyle{fancy}
\fancyhf{}
\rhead{Page \thepage}
\lhead{Nicholas  Ikechukwu - U71641768 - Boston University}
\cfoot{}

\setlength{\parindent}{0pt}

\begin{document}


\begin{center}
    \Large\textbf{Solutions to CS511 Homework 10}
    
    \vspace{0.5cm}
    
    \large Nicholas Ikechukwu - U71641768
    
    \vspace{0.3cm}
    
    \large November 14, 2024
\end{center}



\section*{Exercise 1 Open Lecture Slides 24, “Deductive Closure and First-Order Theories”, page 7:
Carefully answer the question highlighted in green.}

\subsection*{Hint: An appropriate answer should take no more than 4-5 lines. You will find it helpful to read the
preceding pages in the same set of slides. }

\begin{mdframed}
    If $M$ is a relational structure, the first-order theory of $M$ is:
\[
\text{Th}(M) \stackrel{\text{def}}{=} \{ \phi \mid \phi \text{ is a first-order sentence s.t. } M \models \phi \}
\]
Question: Is $\text{Th}(M)$ deductively closed?

\vspace{1em} 
    
\end{mdframed}
\section*{Solution:}

Yes, Th(M) is deductively closed.


\vspace{1em}

\vspace{1em}
To prove this:

\vspace{1em}
Let $\phi \in \overline{\text{Th}(M)}$. Then $\text{Th}(M) \vdash \phi$.

\vspace{1em}
By the Soundness Theorem, $\text{Th}(M) \models \phi$.

\vspace{1em}
Since $M \models \text{Th}(M)$, we have $M \models \phi$.

\vspace{1em}
By definition of $\text{Th}(M)$, $\phi \in \text{Th}(M)$.

\vspace{1em}
Therefore, $\overline{\text{Th}(M)} \subseteq \text{Th}(M)$, so Th(M) is deductively closed.


\newpage



\section*{Exercise 2. Open EML.Chapter 6.pdf : Do Exercise 109 on page 65.}

\subsection*{Hint: There is some reading to do in this exercise, but the answers are straightforward. Correct
answers for parts 1, 2, and 3, are each no more than a single line. An appropriate answer for part 4
invokes Compactness and can be written in three or four lines.}

\vspace{1em}

\begin{mdframed}
    \subsection*{Graph Coloring}

   Hint 1 : Find a way to make use of the following fact: Every finite planar graph is four-colorable.
    (Do not try to prove this fact, which is difficult, but you are allowed to invoke it.)

    \vspace{1em}
    Hint 2 : If M is a planar graph, then every subgraph of M is also planar. A subgraph of M is a
    graph whose vertices are a subset of the vertices of M and whose adjacency relation is a subset
    of the adjacency relation of M restricted to this subset. 
  
\end{mdframed}

\vspace{1em}
In this exercise, I will demonstrate that every infinite planar graph is 4-colorable. The first-order theory of simple undirected graphs can be taken as a set $\Gamma$ of two axioms over signature $\Sigma \stackrel{\text{def}}{=} \{R\}$ consisting of one binary relation symbol, namely:

\[
\Gamma \stackrel{\text{def}}{=} \{\forall x.\forall y. R(x, y) \rightarrow R(y, x), \forall x. \neg R(x, x)\}.
\]

\vspace{1em}
We now expand the signature $\Sigma$ to $\Sigma' = \Sigma \cup \{B, G, P, Y\}$ where $B, G, P$, and $Y$ are unary predicate symbols (for 'blue', 'green', 'purple', and 'yellow').


\subsection*{Part 1}

\vspace{1em}
\begin{mdframed}
Write a first-order sentence $\phi_1$ which, in any $\Sigma'$-structure $M$ satisfying $\Gamma$ (i.e., $M$ is a simple undirected graph),
 asserts "every vertex has at least one of the colors: blue, green, purple, yellow".
\end{mdframed}


\subsection*{Solution}

\[
\phi_1 := \forall x (B(x) \lor G(x) \lor P(x) \lor Y(x))
\]

\newpage



\subsection*{Part 2}

\vspace{1em}
\begin{mdframed}
    Write a first-order sentence $\phi_2$ which, in any $\Sigma'$-structure $M$ satisfying $\Gamma$, 
    asserts "every vertex has at most one color".
\end{mdframed}


\subsection*{Solution}

\[
\phi_2 := \forall x ((B(x) \rightarrow \neg G(x) \land \neg P(x) \land \neg Y(x)) \land (G(x) \rightarrow \neg P(x) \land \neg Y(x)) \land (P(x) \rightarrow \neg Y(x)))
\]

\newpage



\subsection*{Part 3}

\vspace{1em}
\begin{mdframed}
Write a first-order sentence $\phi_3$ which, in any $\Sigma'$-structure $M$ satisfying $\Gamma$, asserts "no two adjacent vertices have the same color".

\end{mdframed}

\subsection*{Solution}

\[
\phi_3 := \forall x \forall y (R(x,y) \rightarrow \neg(B(x) \land B(y)) \land \neg(G(x) \land G(y)) \land \neg(P(x) \land P(y)) \land \neg(Y(x) \land Y(y)))
\]

\newpage

\subsection*{Part 4}


\vspace{1em}
\begin{mdframed}
    Show that if $M$ is an infinite planar graph, i.e.,
    \begin{itemize}
    \item $M$ is a $\Sigma$-structure satisfying $\Gamma$,
    \item the domain of $M$ is infinite, and
    \item $M$ is planar as a graph,
    \end{itemize}

    \vspace{1em}

    then there is a $\Sigma'$-structure $M'$,
    which expands $M$ with four unary relations 

    $B^{M'}, G^{M'}, P^{M'}, and Y^{M'}$, 
    and which satisfies $\phi_1 \land \phi_2 \land \phi_3$, i.e., $M'$ is four-colorable and,
    thus, $M$ is also four-colorable.
    
\end{mdframed}


\subsection*{Solution}

Let $\Delta = \Gamma \cup \{\phi_1, \phi_2, \phi_3\}$. 

\vspace{1em}
Every finite substructure of $M$ is a finite planar graph, hence 4-colorable. 

\vspace{1em}
Thus, every finite subset of $\Delta$ is satisfiable. By Compactness, $\Delta$ is satisfiable.

\vspace{1em}
Therefore, there exists a $\Sigma'$-structure $M'$ expanding $M$ that satisfies $\Delta$, making $M$ four-colorable.
\newpage




\section*{PROBLEM 1 Open EML.Chapter 6.pdf : Do part 1 and part 2 only of Exercise 108 on page
64. You do not need to do part 3 of that exercise.}


\subsection*{Part 1}


\vspace{1em}
\begin{mdframed}
    \textbf{Question:} Give a precise argument in about 5-10 lines for how to systematically generate the countably infinite sequence of $K_{3,3}$ and all its subdivisions, call it $G \stackrel{\text{def}}{=} (G_i \mid i \in \mathbb{N})$, and the countably infinite sequence of $K_5$ and all its subdivisions, call it $G' \stackrel{\text{def}}{=} (G'_i \mid i \in \mathbb{N})$. The first entries in those two sequences are $K_{3,3}$ and $K_5$, i.e., $G_0 \stackrel{\text{def}}{=} K_{3,3}$ and $G'_0 \stackrel{\text{def}}{=} K_5$. It is also useful to define the sequence $G$ so that if $i < j$ then $G_i \leq G_j$, and similarly for the sequence $G'$, i.e., successive entries in $G$ and $G'$ are in order of non-decreasing sizes.

    \vspace{1em}
    \textbf{Hint 1:} It suffices to give an answer for one of the two sequences, say $G$, and to conclude by saying "$G'$ is generated similarly."

    \vspace{1em}
    \textbf{Hint 2:} In the two sequences there are many (though a finite number) subdivisions of the same size. And for the same size, it is possible but quite difficult to omit isomorphic copies; it is much easier to allow isomorphic copies in the two sequences.

    
\end{mdframed}

\subsection*{Solution:}
My precise argument is that, to generate the sequence $G$:

\begin{enumerate}
    \item We start with $G_0 = K_{3,3}$.
    \item For each $i \geq 1$:
        \begin{itemize}
            \item We consider all possible ways to subdivide one edge of $G_{i-1}$.
            \item Add each resulting graph as the next element in the sequence.
            \item If multiple subdivisions result in graphs of the same size, include all of them.
        \end{itemize}
    \item Repeat step 2 indefinitely, ensuring that graphs are added in order of non-decreasing size.
    \item If at any step, all possible single-edge subdivisions of previous graphs have been included, start subdividing two edges, then three, and so on.
\end{enumerate}
This process generates all possible subdivisions of $K_{3,3}$ in a systematic way, allowing for isomorphic copies. $G'$ is generated similarly, starting with $G'_0 = K_5$.


\newpage

\subsection*{Part 2}


\vspace{1em}
\begin{mdframed}
    \textbf{Question:} Let $M \stackrel{\text{def}}{=} (M, R^M)$ be an arbitrary simple graph, $G_i \stackrel{\text{def}}{=} (V_i, E_i)$ an arbitrary subdivision of $K_{3,3}$, and $G'_j \stackrel{\text{def}}{=} (V'_j, E'_j)$ an arbitrary subdivision of $K_5$. Those two subdivisions are entries in the sequences $G$ and $G'$ defined in the preceding part. Write first-order sentences $\phi_i$ and $\phi'_j$ such that if $M \models \phi_i$ (resp. $M \models \phi'_j$), then $G_i$ is a subgraph of $M$ (resp. $G'_j$ is a subgraph of $M$).

    \vspace{1em}
    \textbf{Hint:} You will find it convenient to name the vertices of $G_i$ with an initial segment of the positive integers, i.e., $V_i = \{1, 2, \ldots, n_i\}$ where $n_i$ is the size of $G_i$, and similarly for $G'_j$.
    
\end{mdframed}

\subsection*{Solution:}

Let $V_i = \{1, 2, \ldots, n_i\}$ be the vertices of $G_i$ and $E_i$ its set of edges. We can write $\phi_i$ as:

\[
\phi_i \stackrel{\text{def}}{=} \exists x_1 \ldots \exists x_{n_i} \left(
    \bigwedge_{1 \leq k < l \leq n_i} x_k \not\approx x_l \land
    \bigwedge_{(k,l) \in E_i} R(x_k, x_l)
\right)
\]

Similarly, for $G'_j$ with vertices $V'_j = \{1, 2, \ldots, n'_j\}$ and edges $E'_j$, we can write $\phi'_j$ as:

\[
\phi'_j \stackrel{\text{def}}{=} \exists x_1 \ldots \exists x_{n'_j} \left(
    \bigwedge_{1 \leq k < l \leq n'_j} x_k \not\approx x_l \land
    \bigwedge_{(k,l) \in E'_j} R(x_k, x_l)
\right)
\]

The two first-order sentences assert the existence of distinct vertices in $M$ that are connected according to the edge structure of $G_i$ or $G'_j$, respectively. If $M$ satisfies either of these sentences, it contains the corresponding subdivision as a subgraph.

\newpage



\section*{ON LEAN-4}
\subsection*{Solutions in one file at: 
\url{https://github.com/nich-ikech/CS511-hw-macbeth/blob/main/cs511HwSolutions/hw10/hw10_nicholas_ikechukwu.lean}}

\newpage

\section*{Exercise 3. From Macbeth’s book:}
\section*{Solutions}
\url{https://github.com/nich-ikech/CS511-hw-macbeth/blob/main/cs511HwSolutions/hw10/hw10_nicholas_ikechukwu.lean}

\newpage

\section*{Exercise 4. From Macbeth's book}

\url{https://github.com/nich-ikech/CS511-hw-macbeth/blob/main/cs511HwSolutions/hw10/hw10_nicholas_ikechukwu.lean}

\newpage

\section*{PROBLEM 2. From Macbeth's book}
\section*{Solution}

\url{https://github.com/nich-ikech/CS511-hw-macbeth/blob/main/cs511HwSolutions/hw10/hw10_nicholas_ikechukwu.lean}
\end{document}
