\documentclass{article}
\usepackage[letterpaper]{geometry} % Set the paper size to US Letter
\usepackage{amsmath, amssymb, bussproofs, qtree, booktabs, array, lastpage, fancyhdr}
\usepackage{mdframed}
\usepackage{tcolorbox}
\usepackage{hyperref}
\usepackage{enumitem}



% Define a custom-proof environment

\newenvironment{proof}
{\begin{mdframed}[linewidth=0.5pt]\begin{enumerate}[label=\arabic*.,leftmargin=*]}
{\end{enumerate}\end{mdframed}}

\hypersetup{
  colorlinks=true,
  linkcolor=blue,
  filecolor=magenta,      
  urlcolor=cyan,
  pdftitle={Your Title Here},
  pdfpagemode=FullScreen,
}

\newcommand{\proves}{\vdash}


\pagestyle{fancy}
\fancyhf{}
\rhead{Page \thepage}
\lhead{Nicholas  Ikechukwu - U71641768}
\cfoot{}

\begin{document}


\begin{center}
    \Large\textbf{Solutions to CS511 Homework 03}
    
    \vspace{0.5cm}
    
    \large Nicholas Ikechukwu - U71641768
    
    \vspace{0.3cm}
    
    \large September 25, 2024
\end{center}



\section*{Exercise 1 Go to page 13 in Lecture Slides 09. Your task is to carefully do part 1 of the exercise
on that page.}

\subsection*{Prove by Natural Deduction: $\neg(p \wedge q \wedge r)\rightarrow (\neg p \vee \neg q \vee \neg r)$}
\section*{Solution:}

\begin{enumerate}
    \item $\neg(p \wedge q \wedge r)$  \hfill [Assumption]
    \item $\neg(\neg p \vee \neg q \vee \neg r)$  \hfill [Assumption for contradiction]
    \item $p \wedge q \wedge r$  \hfill [De Morgan's Law applied to line 2]
    \item Contradiction  \hfill [Lines 1 and 3 contradict]
    \item $\neg\neg(\neg p \vee \neg q \vee \neg r)$  \hfill [Negation Introduction, 2-4]
    \item $\neg p \vee \neg q \vee \neg r$  \hfill [Double Negation Elimination, 5]
    \item $\neg(p \wedge q \wedge r) \to (\neg p \vee \neg q \vee \neg r)$  \hfill [Conditional Proof, 1-6]
\end{enumerate}

\newpage

\section*{Exercise 2 Go to page 18 in Lecture Slides 10. Your task is to carefully do part 1 of the exercise
on that page. }
\section*{Use the tableaux method to show the validity of the following more general version of de Morgan's law (4):}

$\phi_1 \equiv \neg(p \land q \land r) \to (\neg p \lor \neg q \lor \neg r)$

\section*{Solution:}
\begin{center}
\begin{tabular}{l l}
1. & $\neg(\neg(p \wedge q \wedge r) \to (\neg p \vee \neg q \vee \neg r))$ \\ 
2. & $p \wedge q \wedge r$ \\ 
3. & $\neg(\neg p \vee \neg q \vee \neg r)$ \\ 
\hline
4. & $p$ \\ 
5. & $q$ \\ 
6. & $r$ \\ 
7. & $\neg p$ \\ 
8. & $\neg q$ \\ 
9. & $\neg r$ \\ 
\hline
10. & Contradiction (from 4 and 7) \\ 
11. & Contradiction (from 5 and 8) \\ 
12. & Contradiction (from 6 and 9) \\ 
\end{tabular}
\end{center}


\begin{center}
\Tree [.{$\neg(\neg(p \land q \land r) \to (\neg p \lor \neg q \lor \neg r))$} 
        [.{$\neg(p \land q \land r)$} ]
        [.{$\neg(\neg p \lor \neg q \lor \neg r)$}
            [.{$p$} ]
            [.{$q$} ]
            [.{$r$} ]
            [.{$\neg p \lor \neg q \lor \neg r$}
                [.{$\neg p$} 
                    [.{X} ] ]
                [.{$\neg q$} 
                    [.{X} ] ]
                [.{$\neg r$} 
                    [.{X} ] ]
            ]
        ]
    ]
\end{center}

\newpage
\section*{PROBLEM 1 There are do parts:
(a) Go to page 13 in Lecture Slides 09 once more. Your task is to carefully do parts 2, 3, and 4
of the exercise on that page.
(b) Go to page 18 in Lecture Slides 10 once more. Your task is to carefully do parts 2, 3, and 4
of the exercise on that page.}

\section*{(a)}

\section*{2. Natural-deduction proof of the most general de Morgan's law}

For $\phi_2 \equiv \neg(p_1 \land \cdots \land p_n) \to (\neg p_1 \lor \cdots \lor \neg p_n)$, where $n \geq 2$:

\begin{proof}
\begin{enumerate}
    \item $\neg(p_1 \land \cdots \land p_n)$ \hfill (Assumption)
    \item $\neg(\neg p_1 \lor \cdots \lor \neg p_n)$ \hfill (Assumption for contradiction)
    \item $p_1 \land \cdots \land p_n$ \hfill (From 2, by De Morgan's law)
    \item Contradiction \hfill (From 1 and 3)
    \item $\neg p_1 \lor \cdots \lor \neg p_n$ \hfill (From 2-4, by Reductio ad Absurdum)
    \item $\neg(p_1 \land \cdots \land p_n) \to (\neg p_1 \lor \cdots \lor \neg p_n)$ \hfill (From 1-5, by Conditional Proof)
\end{enumerate}
\end{proof}

\newpage

\section*{3. Proof length is O(n)}

The natural-deduction proof of $\phi_2$ has a constant number of steps regardless of $n$. The only part that depends on $n$ is the length of the formulas themselves. Therefore, the proof length is O(n).

\newpage

\section*{4. Complexity comparison: Natural-deduction vs. Truth-table}

\begin{itemize}
    \item \textbf{Natural-deduction proof:} As shown above, the proof length is O(n), where n is the number of propositions.
    
    \item \textbf{Truth-table verification:} A truth table for n propositions has $2^n$ rows. Each row requires O(n) operations to compute.
    
    Total complexity: O($n \cdot 2^n$)
\end{itemize}

Comparison: The natural-deduction proof is significantly more efficient, with linear complexity O(n) compared to the exponential complexity O($n \cdot 2^n$) of the truth-table method.

\newpage

\section*{(b)}

\section*{2. Use the tableaux method to show the validity of de Morgan's law (4) in general:}

$\phi_2 \equiv \neg(p_1 \wedge \cdots \wedge p_n) \to (\neg p_1 \vee \cdots \vee \neg p_n)$ where $n \geq 2$.

\begin{tabular}{l l}
1. & $\neg(\neg(p_1 \wedge \cdots \wedge p_n) \to (\neg p_1 \vee \cdots \vee \neg p_n))$ \\
2. & $\neg(p_1 \wedge \cdots \wedge p_n)$ \\
3. & $\neg(\neg p_1 \vee \cdots \vee \neg p_n)$ \\
4. & $p_1$ \\
5. & $p_2$ \\
   & $\vdots$ \\
n+3. & $p_n$ \\
n+4. & $\neg p_1 \vee \neg p_2 \vee \cdots \vee \neg p_n$ \\
\hline
n+5. & $\neg p_1$ \quad X \\
n+6. & $\neg p_2$ \quad X \\
     & $\vdots$ \\
2n+4. & $\neg p_n$ \quad X \\
\end{tabular}

\newpage

\section*{3. Compute the precise size of the tableau (i.e., the number of nodes in the tree underlying the tableau), in Part 2 above, as a function of n (the number of variables).}

The tableau has:
\begin{itemize}
    \item $n+4$ nodes before branching
    \item $n$ branches, each with 1 node
\end{itemize}

Total number of nodes = $(n+4) + n = 2n + 4$
\newpage

\section*{4. Compare the complexity of the tableau proof for $\phi_2$ in Part 2 above with the complexity of the natural-deduction proof of $\phi_2$ and that of the truth-table verification of $\phi_2$. For the latter two procedures, consult Lecture Slides 09.}

\begin{itemize}
    \item[a)] Tableau method: $O(n)$ nodes and steps
    \item[b)] Natural-deduction proof: $O(n)$ steps (as shown in previous lectures)
    \item[c)] Truth-table verification: $O(2^n)$ rows to check all possible combinations
\end{itemize}

Comparison:
\begin{itemize}
    \item The tableau and natural-deduction proofs have linear complexity $O(n)$.
    \item The truth-table verification has exponential complexity $O(2^n)$.
\end{itemize}

For large $n$, the tableau and natural-deduction proofs are significantly more efficient than the truth-table method. The tableau method is comparable in efficiency to the natural-deduction proof for this particular formula.

\newpage
\section*{ON LEAN-4}
\subsection*{Solutions in one file at: 
\url{https://github.com/nich-ikech/CS511-hw-macbeth/blob/main/cs511HwSolutions/hw03/hw03_nicholas_ikechukwu.lean}}

\newpage

\section*{Exercise 3 For each of the three examples in the following three sections of Macbeth’s book, your
task is to remove ‘sorry’ and insert appropriate Lean 4 tactics}
\section*{Solution}
\url{https://github.com/nich-ikech/CS511-hw-macbeth/blob/main/cs511HwSolutions/hw03/hw03_nicholas_ikechukwu.lean}

\newpage

\section*{Exercise 4 For each of the three examples in the following three sections of Macbeth’s book, your
task is to remove ‘sorry’ and insert appropriate Lean 4 tactics.}
\section*{Solution}

\url{https://github.com/nich-ikech/CS511-hw-macbeth/blob/main/cs511HwSolutions/hw03/hw03_nicholas_ikechukwu.lean}

\newpage

\section*{PROBLEM 2 For each of the three examples in the following three sections of Macbeth’s book,
your task is to remove ‘sorry’ and insert appropriate Lean 4 tactics}
\section*{Solution}

\url{https://github.com/nich-ikech/CS511-hw-macbeth/blob/main/cs511HwSolutions/hw03/hw03_nicholas_ikechukwu.lean}
\end{document}
