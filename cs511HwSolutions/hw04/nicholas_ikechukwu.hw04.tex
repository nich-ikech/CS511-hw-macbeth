\documentclass{article}
\usepackage[letterpaper]{geometry} % Set the paper size to US Letter
\usepackage{amsmath, amssymb, bussproofs, qtree, booktabs, array, lastpage, fancyhdr}
\usepackage{mdframed}
\usepackage{tcolorbox}
\usepackage{hyperref}
\usepackage{enumitem}



% Define a custom-proof environment

\newenvironment{proof}
{\begin{mdframed}[linewidth=0.5pt]\begin{enumerate}[label=\arabic*.,leftmargin=*]}
{\end{enumerate}\end{mdframed}}

\hypersetup{
  colorlinks=true,
  linkcolor=blue,
  filecolor=magenta,      
  urlcolor=cyan,
  pdftitle={Your Title Here},
  pdfpagemode=FullScreen,
}

\newcommand{\proves}{\vdash}


\pagestyle{fancy}
\fancyhf{}
\rhead{Page \thepage}
\lhead{Nicholas  Ikechukwu - U71641768}
\cfoot{}

\begin{document}


\begin{center}
    \Large\textbf{Solutions to CS511 Homework 04}
    
    \vspace{0.5cm}
    
    \large Nicholas Ikechukwu - U71641768
    
    \vspace{0.3cm}
    
    \large October 1, 2024
\end{center}



\section*{Exercise 1 [EML.Chapter 1.pdf, page 15-16]: Do part 1 of Exercise 27.}

\section*{Part 1 of Exercise 27 (Queens Problem)}
\section*{Solution:}

The wff $\psi_n$ for the n-Queens Problem can be written as:

\[ \psi_n = \psi_{\text{row}_n} \wedge \psi_{\text{col}_n} \wedge \psi_{\text{diag1}_n} \wedge \psi_{\text{diag2}_n} \]

Where:

\begin{enumerate}
    \item[$a)$] $\psi_{\text{row}_n}$ (exactly one queen in each row):
    \[ \bigwedge_{i=1}^n \left(\bigvee_{j=1}^n q_{i,j} \wedge \bigwedge_{1 \leq j < k \leq n} (\neg q_{i,j} \vee \neg q_{i,k})\right) \]

    \item[$b)$] $\psi_{\text{col}_n}$ (exactly one queen in each column):
    \[ \bigwedge_{j=1}^n \left(\bigvee_{i=1}^n q_{i,j} \wedge \bigwedge_{1 \leq i < k \leq n} (\neg q_{i,j} \vee \neg q_{k,j})\right) \]

    \item[$c)$] $\psi_{\text{diag1}_n}$ (at most one queen in each diagonal):
    \[ \bigwedge_{1 \leq i_1 < i_2 \leq n} \bigwedge_{1 \leq j_1 < j_2 \leq n} ((i_1 - j_1 = i_2 - j_2) \rightarrow (\neg q_{i_1,j_1} \vee \neg q_{i_2,j_2})) \]

    \item[$d)$] $\psi_{\text{diag2}_n}$ (at most one queen in each antidiagonal):
    \[ \bigwedge_{1 \leq i_1 < i_2 \leq n} \bigwedge_{1 \leq j_1 < j_2 \leq n} ((i_1 + j_1 = i_2 + j_2) \rightarrow (\neg q_{i_1,j_1} \vee \neg q_{i_2,j_2})) \]
\end{enumerate}

\textbf{Justification:}
\begin{itemize}
    \item $\psi_{\text{row}_n}$ ensures each row has exactly one queen by requiring at least one queen per row and no two queens in the same row.
    \item $\psi_{\text{col}_n}$ does the same for columns.
    \item $\psi_{\text{diag1}_n}$ prevents more than one queen on any diagonal by ensuring no two queens share the same $i-j$ difference.
    \item $\psi_{\text{diag2}_n}$ prevents more than one queen on any antidiagonal by ensuring no two queens share the same $i+j$ sum.
\end{itemize}

The conjunction of these four conditions fully specifies the n-Queens Problem, ensuring a valid placement of $n$ queens on an $n \times n$ board where no two queens can attack each other.

\newpage

\section*{Exercise 2 [Lecture Slides 13, page 19]: Do the exercise on that page}

\section*{Formalized Simplified Definition of Substitution}
\section*{Solution:}

\subsection*{ BNF Definition (Modified) with Free Variables:}

\begin{verbatim}
<wff> ::= T | F | <var> | ~<wff> | (<wff> & <wff>) | (<wff> | <wff>) | 
          (<wff> -> <wff>) | A<var><wff> | E<var><wff>
\end{verbatim}

Where:
\begin{itemize}
    \item For $\phi = \top$ or $\bot$, $FV(\phi) = \emptyset$
    \item For $\phi = x$ (a variable), $FV(\phi) = \{x\}$
    \item For $\phi = \neg\psi$, $FV(\phi) = FV(\psi)$
    \item For $\phi = (\psi_1 \star \psi_2)$ where $\star \in \{\wedge, \vee, \rightarrow\}$, $FV(\phi) = FV(\psi_1) \cup FV(\psi_2)$
    \item For $\phi = Qx \psi$ where $Q \in \{\forall, \exists\}$, $FV(\phi) = FV(\psi) \setminus \{x\}$
\end{itemize}

Some additional constraints:
\begin{enumerate}
    \item In any wff, there is at most one binding occurrence for each variable.
    \item If $x \in FV(\phi)$, then $x$ does not occur bound in $\phi$.
\end{enumerate}

\subsection*{Simplified Substitution Definition:}

Given the constraints, we can simplify the substitution definition as follows:

\[
\phi[u/x] =
\begin{cases}
    \phi                   & \text{if } \phi = \top \text{ or } \bot \\
    u                      & \text{if } \phi = x \\
    \phi                   & \text{if } \phi = y \text{ and } x \neq y \\
    \neg(\psi[u/x])        & \text{if } \phi = \neg\psi \\
    \psi_1[u/x] \star \psi_2[u/x]  & \text{if } \phi = \psi_1 \star \psi_2 \text{ and } \star \in \{\wedge, \vee, \rightarrow\} \\
    Qy (\psi[u/x])         & \text{if } \phi = Qy \psi, Q \in \{\forall, \exists\}, \text{ and } x \neq y \\
    \phi                   & \text{if } \phi = Qx \psi \text{ and } Q \in \{\forall, \exists\}
\end{cases}
\]

Where:
\begin{itemize}
    \item $u$ is $\top$, $\bot$, or a variable
    \item The substitution is only defined if $u$ is substitutable for $x$ in $\phi$, which is always true under our constraints.
\end{itemize}

\subsection*{Justification for the  Simplification above:}

\begin{itemize}
    \item The case ``$\phi$ if $\phi = Qy \phi', Q \in \{\forall, \exists\}, x = y$'' is now covered by the last case ``$\phi$ if $\phi = Qx \psi$ and $Q \in \{\forall, \exists\}$'' because there's at most one binding occurrence for each variable.

    \item We don't need to check if $u$ is substitutable for $x$ in $\phi$ in the quantifier case because:
    \begin{enumerate}
        \item There's at most one binding occurrence for each variable.
        \item A variable cannot have both free and bound occurrences.
    \end{enumerate}
    These conditions ensure that there's no risk of variable capture during substitution.

    \item The case for variables is simplified because we don't need to distinguish between free and bound occurrences of variables anymore.
\end{itemize}



\newpage
\section*{PROBLEM 1 [EML.Chapter 1.pdf, page 15-16]: Do parts 2, 3, and 4 of Exercise 27. .}
\section*{n-Queens Problem (continued)}

\subsection*{(2). Infinite Chessboard Argument}

The argument for the Infinite Queens Problem is flawed for the following reason:

While $\Gamma = \{\psi_n | n \geq 4\}$ is indeed finitely satisfiable (as any finite subset corresponds to a finite n-Queens problem which has a solution), the satisfaction of all $\psi_n$ for $n \geq 4$ does not necessarily imply a solution to the Infinite Queens Problem.

The key issue is that each $\psi_n$ only constrains a finite $n \times n$ portion of the infinite board. A model satisfying all $\psi_n$ might place queens in a way that satisfies each finite constraint, but could potentially place infinitely many queens in some rows, columns, or diagonals when considering the entire infinite board.

In other words, the conditions (a), (b), (c), and (d) for all $n \geq 4$ do not fully capture the constraints of the Infinite Queens Problem, which requires that no two queens attack each other on the entire infinite board.

\newpage

\subsection*{(3). Defining $\Theta$}

Let's define $\Theta = \{\theta_k | k \geq 1\}$ as follows:

\begin{align*}
\theta_k = \bigwedge_{i=1}^k \bigwedge_{j=1}^k &\left(q_{i,j} \leftrightarrow (i+j \equiv k+1 \pmod{k})\right) \\
&\wedge \bigwedge_{i=k+1}^\infty \bigwedge_{j=k+1}^\infty \neg q_{i,j}
\end{align*}

This definition satisfies the required conditions:

(a) Each $\theta_k$ defines a solution to the $k$-Queens Problem by placing queens on the $k \times k$ board in positions where $i+j \equiv k+1 \pmod{k}$, and no queens elsewhere.

(b) For $k' > k$, $\theta_{k'}$ defines a solution to a larger Queens Problem than $\theta_k$.

(c) Any finite subset of $\Theta$ is satisfiable, as each $\theta_k$ defines a distinct, valid queen placement on a finite portion of the board.
\newpage


\subsection*{(4). Solution to the Infinite Queens Problem}

Using the Compactness Theorem for Propositional Logic, we can prove that the Infinite Queens Problem has a solution:

a. We have shown that every finite subset of $\Theta$ is satisfiable.

b. By the Compactness Theorem, $\Theta$ itself is satisfiable.

c. Let $v$ be a truth assignment that satisfies all wffs in $\Theta$.

d. Define a queen placement on the infinite board as follows:
   Place a queen at position $(i,j)$ if and only if $v(q_{i,j}) = \text{true}$.

e. This placement is a valid solution to the Infinite Queens Problem because:
   \begin{itemize}
     \item For each $k$, the placement satisfies $\theta_k$, ensuring a valid $k$-Queens solution in the $k \times k$ subboard.
     \item As $k$ increases, larger portions of the board are covered with valid queen placements.
     \item In the limit, this results in a valid placement for the entire infinite board.
   \end{itemize}

Therefore, we have rigorously shown that the Infinite Queens Problem has a solution.

\newpage







\section*{ON LEAN-4}
\subsection*{Solutions in one file at: 
\url{https://github.com/nich-ikech/CS511-hw-macbeth/blob/main/cs511HwSolutions/hw04/hw04_nicholas_ikechukwu.lean}}

\newpage

\section*{Exercise 3. Hint: These should be easy if you read the book. Use existential quantifiers.}
\section*{Solution}
\url{https://github.com/nich-ikech/CS511-hw-macbeth/blob/main/cs511HwSolutions/hw04/hw04_nicholas_ikechukwu.lean}

\newpage

\section*{Exercise 4. Hint: These use existential and universal quantifiers. The existential quantifiers are used in both
context and goal, but universal quantifiers only in context.}

\url{https://github.com/nich-ikech/CS511-hw-macbeth/blob/main/cs511HwSolutions/hw04/hw04_nicholas_ikechukwu.lean}

\newpage

\section*{PROBLEM 2. From Macbeth’s book: Hint: The first will be hard unless you use the lemma listed in the book. The other two involve some
computation, but they should be easy if you make use of scratch paper while solving. The very last
one shows that order matters when rewriting and is similar otherwise.}
\section*{Solution}

\url{https://github.com/nich-ikech/CS511-hw-macbeth/blob/main/cs511HwSolutions/hw04/hw04_nicholas_ikechukwu.lean}
\end{document}
