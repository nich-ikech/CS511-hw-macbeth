\documentclass{article}
\usepackage[letterpaper]{geometry} % Set the paper size to US Letter
\usepackage{amsmath, amssymb, bussproofs, qtree, booktabs, array, lastpage, fancyhdr}
\usepackage{mdframed}
\usepackage{tcolorbox}
\usepackage{hyperref}
\usepackage{enumitem}
\usepackage{stmaryrd}



% \defeqine a custom-proof environment

\newenvironment{proof}
{\begin{mdframed}[linewidth=0.5pt]\begin{enumerate}[label=\arabic*.,leftmargin=*]}
{\end{enumerate}\end{mdframed}}

\hypersetup{
  colorlinks=true,
  linkcolor=blue,
  filecolor=magenta,      
  urlcolor=cyan,
  pdftitle={Your Title Here},
  pdfpagemode=FullScreen,
}
 

% Add the new command here
\newcommand{\defeq}{\stackrel{\text{def}}{=}}
\newcommand{\proves}{\vdash}


\pagestyle{fancy}
\fancyhf{}
\rhead{Page \thepage}
\lhead{Nicholas  Ikechukwu - U71641768 - Boston University}
\cfoot{}

\setlength{\parindent}{0pt}

\begin{document}


\begin{center}
    \Large\textbf{Solutions to CS511 Homework 11}
    
    \vspace{0.5cm}
    
    \large Nicholas Ikechukwu - U71641768
    
    \vspace{0.3cm}
    
    \large November 21, 2024
\end{center}



\section*{Exercise 1. Open [LCS, page 164]: Do Exercise 2.4.12, parts (a) and (b) only.}



\begin{mdframed}
\vspace{1em}
    \textbf{Question Predicate Logic Formulas}: For each of the formulas of predicate logic below, either find a model which
    does not satisfy it, or prove it is valid:
\vspace{1em}
\end{mdframed}

\section*{Solutions: }

\subsection*{(a) $(\forall x \forall y (S(x, y) \rightarrow S(y, x))) \rightarrow (\forall x \neg S(x, x))$}

\vspace{1em}
\textbf{Answer:} Formula is not valid. Let us find a counterexample:

\vspace{1em}

Let the domain be $\{a\}$ and interpret $S$ as $S = \{(a,a)\}$.

\vspace{1em}

In this model:
\begin{itemize}
    \item $\forall x \forall y (S(x, y) \rightarrow S(y, x))$ is true because $S$ is symmetric.
    \item $\forall x \neg S(x, x)$ is false because $S(a,a)$ is true.
\end{itemize}

This shows that the antecedent is true while the consequent is false, making the implication false.
\vspace{1em}


\newpage

\subsection*{(b) $\exists y ((\forall x P(x)) \rightarrow P(y))$}

\vspace{1em}
\textbf{Answer:} I believe the Formula here is also valid. We can prove it by cases:

\vspace{1em}
Case 1: If $\forall x P(x)$ is true, then $P(y)$ is true for any $y$, so the implication is true.

\vspace{1em}
Case 2: If $\forall x P(x)$ is false, then there exists some element $a$ in the domain for which $P(a)$ is false. 

\vspace{1em}
Choose $y = a$. Then both the antecedent and consequent of the implication are false, making the implication true.

\vspace{1em}
In both cases, we can find a $y$ that makes the formula true, so $\exists y ((\forall x P(x)) \rightarrow P(y))$ is always true.


\newpage


\section*{Exercise 2. Open [LCS, page 164]: Do Exercise 2.4.12, parts (c) and (d) only.}

\begin{mdframed}
    \vspace{1em}
        \textbf{Question Predicate Logic Formulas}: For each of the formulas of predicate logic below, either find a model which
        does not satisfy it, or prove it is valid:
    \vspace{1em}
    \end{mdframed}
    
\section*{Solutions: }

\vspace{1em}

\subsection*{(c) $(\forall x (P(x) \rightarrow \exists y Q(y))) \rightarrow (\forall x \exists y (P(x) \rightarrow Q(y)))$}


\textbf{Answer:} I suggest that this formula is also valid. Let's prove it by contradiction:

\vspace{1em}
Let's assume the antecedent is true and the consequent is false.

\vspace{1em}

That is:

\vspace{1em}
1) $\forall x (P(x) \rightarrow \exists y Q(y))$ is true

\vspace{1em}
2) $\forall x \exists y (P(x) \rightarrow Q(y))$ is false

\vspace{1em}
From the consequent, (2), there must be some $a$ such that $\forall y (P(a) \rightarrow Q(y))$ is false.

\vspace{1em}
We can tell that $P(a)$ is true and $\forall y \neg Q(y)$ is true.

\vspace{1em}
However, from (1), we know that $P(a) \rightarrow \exists y Q(y)$ is true.

\vspace{1em}
Since $P(a)$ is true, $\exists y Q(y)$ must be true.

\vspace{1em}
It's clear that this contradicts $\forall y \neg Q(y)$.

\vspace{1em}
Therefore, what we assumed, must be false, and the formula is valid.

\newpage


\subsection*{(d) $(\forall x \exists y (P(x) \rightarrow Q(y))) \rightarrow (\forall x (P(x) \rightarrow \exists y Q(y)))$}

\vspace{1em}

\vspace{1em}
\textbf{Answer:} This formula is valid. We can actually prove it directly:

\vspace{1em}
Let's assume the antecedent is true: $\forall x \exists y (P(x) \rightarrow Q(y))$

\vspace{1em}
Now, consider any arbitrary $x$:

\vspace{1em}
1) If $P(x)$ is false, then $P(x) \rightarrow \exists y Q(y)$ is trivially true.

\vspace{1em}
2) If $P(x)$ is true, then from our assumption, there exists a $y$ such that $Q(y)$ is true.

\vspace{1em}
This allows $\exists y Q(y)$ to be true, and thus $P(x) \rightarrow \exists y Q(y)$ is true.

\vspace{1em}
In both cases, $P(x) \rightarrow \exists y Q(y)$ is true for any $x$.

\vspace{1em}
Therefore, $\forall x (P(x) \rightarrow \exists y Q(y))$ is true, making the entire implication true.

\vspace{1em}
Hence, the formula is valid.

\newpage
\section*{PROBLEM 1. Open EML.Chapter 6.pdf : Do Exercise 100 on page 62.}


\begin{mdframed}
    \subsection*{Question}

    Let $R$ be a unary relation symbol and consider the following inductively defined translation from PL to FOL, $\langle \rangle : \text{WFF}_\text{PL}(P) \rightarrow \text{WFF}_\text{FOL}(\{R\}, P)$:
    
    \begin{align*}
    \langle p_i \rangle &\overset{\text{def}}{=} R(p_i) \text{ for every } p_i \in P, \\
    \langle \bot \rangle &\overset{\text{def}}{=} \bot, \\
    \langle \top \rangle &\overset{\text{def}}{=} \top, \\
    \langle \neg\phi \rangle &\overset{\text{def}}{=} \neg\langle\phi\rangle, \\
    \langle \phi \diamond \psi \rangle &\overset{\text{def}}{=} \langle\phi\rangle \diamond \langle\psi\rangle \text{ where } \diamond \in \{\wedge, \vee, \rightarrow\}.
    \end{align*}
    
    Give a rigorous argument, using structural induction, to establish the following assertions:
    
    \begin{enumerate}
    \item $\phi$ is satisfiable in the sense of PL iff $\langle\phi\rangle$ is satisfiable in the sense of FOL.
    \item $\phi$ is valid in the sense of PL iff $\langle\phi\rangle$ is valid in the sense of FOL.
    \end{enumerate}
    
    Note that the assertions in parts 1 and 2 involve two implications because of "iff" and each of the two implications have to be proved separately.
    
    Someone suggested the following translation
    $\llbracket \rrbracket : \text{WFF}_\text{PL}(P) \rightarrow \text{WFF}_\text{FOL}(\{R\}, P)$ instead of $\langle \rangle$:
    
    \[
    \llbracket\phi\rrbracket \overset{\text{def}}{=} \langle\phi\rangle \wedge (\exists x R(x) \wedge (\exists x \neg R(x))
    \]
    
    where $x$ is a fresh first-order variable not in $P$. Although the translation $\llbracket \rrbracket$ can be used instead of $\langle \rangle$, and was presented as being "better" than $\langle \rangle$, give a rigorous argument for the following:
    
    \begin{enumerate}
    \setcounter{enumi}{2}
    \item The added requirement in the translation $\llbracket \rrbracket$, expressed by $(\exists x R(x) \wedge (\exists x \neg R(x))$, is not necessary for correct proofs of the assertions in parts 1 and 2.
    \end{enumerate}
\end{mdframed}

\section*{First-Order Logic: Interpretation of Propositional Logic}



\subsection*{Answer}

\subsubsection*{1. Satisfiability}

Via structural induction we will prove by that $\phi$ is satisfiable in PL iff $\langle\phi\rangle$ is 
satisfiable in FOL.

\vspace{1em}
The base cases:
\begin{itemize}
\item For $p_i \in P$: we know that $p_i$ is satisfiable in PL iff there exists a valuation $v$ such that $v(p_i) = \text{true}$. This is equivalent to the existence of a first-order structure $\mathcal{A}$ and variable assignment $s$ such that $\mathcal{A} \models_s R(p_i)$, which is the definition of satisfiability for $\langle p_i \rangle$ in FOL.
\item $\bot$ is not satisfiable in PL and $\langle\bot\rangle = \bot$ is not satisfiable in FOL.
\item $\top$ is satisfiable in PL and $\langle\top\rangle = \top$ is satisfiable in FOL.
\end{itemize}

Inductive step:
Let's assume that for $\phi$ and $\psi$, the statement holds.

\begin{itemize}
\item $\neg\phi$ is satisfiable in PL iff $\phi$ is not valid in PL. By the induction hypothesis, this is equivalent to $\langle\phi\rangle$ not being valid in FOL, which is equivalent to $\neg\langle\phi\rangle = \langle\neg\phi\rangle$ being satisfiable in FOL.

\item For $\diamond \in \{\wedge, \vee, \rightarrow\}$, $\phi \diamond \psi$ is satisfiable in PL iff there exists a valuation satisfying the truth table for $\diamond$. By the induction hypothesis, this is equivalent to the existence of a first-order structure and variable assignment satisfying $\langle\phi\rangle \diamond \langle\psi\rangle = \langle\phi \diamond \psi\rangle$ in FOL.
\end{itemize}


\newpage
\subsubsection*{2. Validity}

The proof for validity follows a similar structure to the satisfiability proof, using the fact that a formula is valid iff its negation is not satisfiable.


\newpage
\subsubsection*{3. The Unnecessary Additional Requirement}

I think that the added requirement $(\exists x R(x) \wedge (\exists x \neg R(x))$ in the translation $\llbracket \rrbracket$
is not necessary for correct proofs of the assertions in parts 1 and 2 for the following reasons:

\begin{itemize}
\item The proofs for parts 1 and 2 rely on the structural correspondence between PL formulas and their FOL translations,
 which is preserved in $\langle \rangle$ and not affected by the additional conjunct in $\llbracket \rrbracket$.
\item This additional conjunct is always satisfiable in FOL, as it merely asserts the existence of at least one element in
 the domain for which $R$ is true and at least one for which $R$ is false.
\item The satisfiability and validity of $\langle\phi\rangle$ in FOL are independent of this additional conjunct.
 If $\langle\phi\rangle$ is satisfiable (or valid), then $\llbracket\phi\rrbracket$ is satisfiable (or valid) with the same model extended to satisfy the additional conjunct. Conversely, if $\llbracket\phi\rrbracket$ is satisfiable (or valid), then $\langle\phi\rangle$ must also be satisfiable (or valid) in the same model.
\end{itemize}

Therefore, while $\llbracket \rrbracket$ may provide other benefits, 
it barely has an effect on the correctness of the proofs for satisfiability and validity equivalence between PL and FOL translations.
\newpage



\section*{ON LEAN-4}
\subsection*{Solutions in one file at: 
\url{https://github.com/nich-ikech/CS511-hw-macbeth/blob/main/cs511HwSolutions/hw11/hw11_nicholas_ikechukwu.lean}}

\newpage

\section*{Exercise 3. From Macbeth’s book:}
\section*{Solutions}
\url{https://github.com/nich-ikech/CS511-hw-macbeth/blob/main/cs511HwSolutions/hw11/hw11_nicholas_ikechukwu.lean}

\newpage

\section*{Exercise 4. From Macbeth's book}

\url{https://github.com/nich-ikech/CS511-hw-macbeth/blob/main/cs511HwSolutions/hw11/hw11_nicholas_ikechukwu.lean}

\newpage

\section*{PROBLEM 2. From Macbeth's book}
\section*{Solution}

\url{https://github.com/nich-ikech/CS511-hw-macbeth/blob/main/cs511HwSolutions/hw11/hw11_nicholas_ikechukwu.lean}
\end{document}
