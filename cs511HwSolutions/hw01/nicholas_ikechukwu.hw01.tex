\documentclass{article}
\usepackage[letterpaper]{geometry} % Set the paper size to US Letter
\usepackage{amsmath, amssymb, bussproofs, qtree, booktabs, array, lastpage, fancyhdr}
\usepackage{mdframed}
\usepackage{tcolorbox}
\usepackage{hyperref}

\hypersetup{
  colorlinks=true,
  linkcolor=blue,
  filecolor=magenta,      
  urlcolor=cyan,
  pdftitle={Your Title Here},
  pdfpagemode=FullScreen,
}

\newcommand{\proves}{\vdash}

% Define a custom proof environment
\newenvironment{proof}{\par\noindent\textbf{Proof:}\quad}{\hfill$\square$\par\vspace{1em}}
\pagestyle{fancy}
\fancyhf{}
\rhead{Page \thepage}
\lhead{Nicholas Ikechukwu - U71641768}
\cfoot{}

\begin{document}


\begin{center}
    \Large\textbf{Solutions to CS511 Homework 01}
    
    \vspace{0.5cm}
    
    \large Nicholas Ikechukwu - U71641768
    
    \vspace{0.3cm}
    
    \large September 11, 2024
\end{center}


% Exercise 1 [LCS, page 79]: Exercise 1.2.1. Do parts (h), (i), and (j). 
% Exercise 2 [LCS, page 84]: Exercise 1.4.2. Do parts (g), (h), and (i). 
% PROBLEM 1 [LCS, page 87]: Exercise 1.5.3. Do parts (b) and (c)


\section*{Exercise 1 [LCS, page 79]: Exercise 1.2.1. parts (h), (i), and (j)}
Prove the validity of the following sequents:


\section*{Sequent Proofs}

\subsection*{(h) $p \proves (p \rightarrow q) \rightarrow q$}

\begin{proof}
We will use natural deduction to prove this sequent.

\begin{tabular}{r@{ }l@{\qquad}l}
1. & $p$ & (Premise) \\
\multicolumn{3}{l}{\fbox{\begin{minipage}{\dimexpr\linewidth-2\fboxsep-2\fboxrule\relax}
2. \quad $p \rightarrow q$ \qquad (Assumption) \\
3. \quad $q$ \qquad (Modus Ponens 1, 2)
\end{minipage}}} \\
4. & $(p \rightarrow q) \rightarrow q$ & ($\rightarrow$-Introduction 2-3)
\end{tabular}\\\\
 
This proof is valid because:
\begin{itemize}
    \item We start with the premise $p$.
    \item We assume $p \rightarrow q$ (for $\rightarrow$-Introduction).
    \item We use Modus Ponens with lines 1 and 2 to derive $q$.
    \item We conclude $(p \rightarrow q) \rightarrow q$ by $\rightarrow$-Introduction, discharging the assumption in line 2.
\end{itemize}
\end{proof}

\newpage
\subsection*{(i) $(p \rightarrow r) \wedge (q \rightarrow r) \proves p \wedge q \rightarrow r$}


\begin{proof}
We will use natural deduction to prove this sequent.

\begin{tabular}{r@{ }l@{\qquad}l}
1. & $(p \rightarrow r) \wedge (q \rightarrow r)$ & (Premise) \\
2. & $p \rightarrow r$ & ($\wedge$-Elimination 1) \\
3. & $q \rightarrow r$ & ($\wedge$-Elimination 1) \\
\multicolumn{3}{l}{\fbox{\begin{minipage}{\dimexpr\linewidth-2\fboxsep-2\fboxrule\relax}
4. \quad $p \wedge q$ \qquad (Assumption) \\
5. \quad $p$ \qquad ($\wedge$-Elimination 4) \\
6. \quad $q$ \qquad ($\wedge$-Elimination 4) \\
7. \quad $r$ \qquad (Modus Ponens 2, 5)
\end{minipage}}} \\
8. & $p \wedge q \rightarrow r$ & ($\rightarrow$-Introduction 4-7)
\end{tabular}\\\\

This proof is valid because:
\begin{itemize}
    \item We start with the premise $(p \rightarrow r) \wedge (q \rightarrow r)$.
    \item We use $\wedge$-Elimination to derive $p \rightarrow r$ and $q \rightarrow r$.
    \item We assume $p \wedge q$ (for $\rightarrow$-Introduction).
    \item We use $\wedge$-Elimination to derive $p$ and $q$ separately.
    \item We use Modus Ponens with $p \rightarrow r$ and $p$ to derive $r$.
    \item We conclude $p \wedge q \rightarrow r$ by $\rightarrow$-Introduction, discharging the assumption in line 4.
\end{itemize}
\end{proof}

\newpage
\subsection*{(j) $q \rightarrow r \proves (p \rightarrow q) \rightarrow (p \rightarrow r)$}

\begin{proof}
We will use natural deduction to prove this sequent.

\begin{tabular}{r@{ }l@{\qquad}l}
1. & $q \rightarrow r$ & (Premise) \\
\multicolumn{3}{l}{\fbox{\begin{minipage}{\dimexpr\linewidth-2\fboxsep-2\fboxrule\relax}
2. \quad $p \rightarrow q$ \qquad (Assumption) \\
\fbox{\begin{minipage}{\dimexpr\linewidth-2\fboxsep-2\fboxrule\relax}
3. \quad \quad $p$ \qquad (Assumption) \\
4. \quad \quad $q$ \qquad (Modus Ponens 2, 3) \\
5. \quad \quad $r$ \qquad (Modus Ponens 1, 4)
\end{minipage}} \\
6. \quad $p \rightarrow r$ \qquad ($\rightarrow$-Introduction 3-5)
\end{minipage}}} \\
7. & $(p \rightarrow q) \rightarrow (p \rightarrow r)$ & ($\rightarrow$-Introduction 2-6)
\end{tabular}\\\\

This proof is valid because:
\begin{itemize}
    \item We start with the premise $q \rightarrow r$.
    \item We assume $p \rightarrow q$ (for outer $\rightarrow$-Introduction).
    \item We assume $p$ (for inner $\rightarrow$-Introduction).
    \item We use Modus Ponens twice to derive $r$.
    \item We conclude $p \rightarrow r$ by $\rightarrow$-Introduction, discharging the assumption in line 3.
    \item We conclude $(p \rightarrow q) \rightarrow (p \rightarrow r)$ by $\rightarrow$-Introduction, discharging the assumption in line 2.
\end{itemize}
\end{proof}


\newpage
\section*{Exercise 2 [LCS, page 84]: Exercise 1.4.2. parts (g), (h), and (i).}
\textbf{Compute the complete truth table of the formulae:}

\section*{Complete Truth Tables}

\subsection*{(g) $((p \rightarrow q) \rightarrow p) \rightarrow p$}

\begin{table}[h]
\centering
\begin{tabular}{ccc|ccc|c}
\toprule
$p$ & $q$ & $p \rightarrow q$ & $(p \rightarrow q) \rightarrow p$ & $((p \rightarrow q) \rightarrow p) \rightarrow p$ \\
\midrule
T & T & T & T & T \\
T & F & F & T & T \\
F & T & T & F & T \\
F & F & T & F & T \\
\bottomrule
\end{tabular}
\caption{Truth table for $((p \rightarrow q) \rightarrow p) \rightarrow p$}
\end{table}

\newpage
\subsection*{(h) $((p \vee q) \rightarrow r) \rightarrow ((p \rightarrow r) \vee (q \rightarrow r))$}

\begin{table}[h]
\centering
\begin{tabular}{ccc|ccccc|c}
\toprule
$p$ & $q$ & $r$ & $p \vee q$ & $(p \vee q) \rightarrow r$ & $p \rightarrow r$ & $q \rightarrow r$ & $(p \rightarrow r) \vee (q \rightarrow r)$ & $((p \vee q) \rightarrow r) \rightarrow ((p \rightarrow r) \vee (q \rightarrow r))$ \\
\midrule
T & T & T & T & T & T & T & T & T \\
T & T & F & T & F & F & F & F & T \\
T & F & T & T & T & T & T & T & T \\
T & F & F & T & F & F & T & T & T \\
F & T & T & T & T & T & T & T & T \\
F & T & F & T & F & T & F & T & T \\
F & F & T & F & T & T & T & T & T \\
F & F & F & F & T & T & T & T & T \\
\bottomrule
\end{tabular}
\caption{Truth table for $((p \vee q) \rightarrow r) \rightarrow ((p \rightarrow r) \vee (q \rightarrow r))$}
\end{table}

\newpage
\subsection*{(i) $(p \rightarrow q) \rightarrow (\neg p \rightarrow \neg q)$}

\begin{table}[h]
\centering
\begin{tabular}{cc|cccc|c}
\toprule
$p$ & $q$ & $p \rightarrow q$ & $\neg p$ & $\neg q$ & $\neg p \rightarrow \neg q$ & $(p \rightarrow q) \rightarrow (\neg p \rightarrow \neg q)$ \\
\midrule
T & T & T & F & F & T & T \\
T & F & F & F & T & T & T \\
F & T & T & T & F & F & F \\
F & F & T & T & T & T & T \\
\bottomrule
\end{tabular}
\caption{Truth table for $(p \rightarrow q) \rightarrow (\neg p \rightarrow \neg q)$}
\end{table}


\medskip

\noindent

\newpage
\section*{PROBLEM 1 [LCS, page 87]: Exercise 1.5.3. parts (b) and (c).}

\section*{3. Adequate Set of Connectives}

\subsection*{(b) Showing that if $C \subseteq \{\neg, \wedge, \vee, \rightarrow, \bot\}$ is adequate, then $\neg \in C$ or $\bot \in C$}

\textbf{Proof by contradiction:}

Assume that $C$ is adequate and that neither $\neg \in C$ nor $\bot \in C$. 

Let $v$ be a valuation that assigns T to all atomic propositions. Consider any formula $\phi$ constructed using only connectives from $C$. We will prove by structural induction that $v(\phi) = \text{T}$ for all such $\phi$.

Base case: 
\begin{itemize}
    \item If $\phi$ is an atomic proposition, then $v(\phi) = \text{T}$ by definition of $v$.
\end{itemize}

Inductive step:
\begin{itemize}
    \item If $\phi = \psi \wedge \chi$, then $v(\phi) = v(\psi) \wedge v(\chi) = \text{T} \wedge \text{T} = \text{T}$
    \item If $\phi = \psi \vee \chi$, then $v(\phi) = v(\psi) \vee v(\chi) = \text{T} \vee \text{T} = \text{T}$
    \item If $\phi = \psi \rightarrow \chi$, then $v(\phi) = v(\psi) \rightarrow v(\chi) = \text{T} \rightarrow \text{T} = \text{T}$
\end{itemize}

Therefore, any formula constructed using only connectives from $C$ will always evaluate to T under valuation $v$.

However, the formula $\bot$ (false) should always evaluate to F under any valuation. Since $C$ is assumed to be adequate, it must be able to express $\bot$, which is impossible given our proof.

This contradiction shows that our initial assumption must be false. Therefore, if $C$ is adequate, then $\neg \in C$ or $\bot \in C$.

\newpage
\subsection*{(c) Is $\{\leftrightarrow, \neg\}$ adequate?}

\textbf{Claim:} The set $\{\leftrightarrow, \neg\}$ is adequate for propositional logic.

\textbf{Proof:}

To prove adequacy, we need to show that we can express all other connectives using only $\leftrightarrow$ and $\neg$. We'll do this by providing equivalent formulas for $\wedge$, $\vee$, and $\rightarrow$.

1. Expressing $\wedge$:
   $p \wedge q \equiv \neg(p \leftrightarrow \neg q)$
   
   Proof of equivalence:
   \begin{itemize}
       \item When $p$ and $q$ are both T, $\neg q$ is F, so $p \leftrightarrow \neg q$ is F, and $\neg(p \leftrightarrow \neg q)$ is T.
       \item In all other cases, either $p$ is F or $q$ is F (or both), so $p \leftrightarrow \neg q$ is T, and $\neg(p \leftrightarrow \neg q)$ is F.
   \end{itemize}

2. Expressing $\vee$:
   $p \vee q \equiv (p \leftrightarrow q) \leftrightarrow (p \leftrightarrow p)$
   
   Proof of equivalence:
   \begin{itemize}
       \item When either $p$ or $q$ (or both) are T, $p \leftrightarrow q$ is not equivalent to $p \leftrightarrow p$ (which is always T), so the overall expression is T.
       \item When both $p$ and $q$ are F, $p \leftrightarrow q$ is T, which is equivalent to $p \leftrightarrow p$, so the overall expression is F.
   \end{itemize}

3. Expressing $\rightarrow$:
   $p \rightarrow q \equiv \neg p \leftrightarrow (p \leftrightarrow q)$
   
   Proof of equivalence:
   \begin{itemize}
       \item When $p$ is T and $q$ is F, $\neg p$ is F, $p \leftrightarrow q$ is F, so $\neg p \leftrightarrow (p \leftrightarrow q)$ is T.
       \item In all other cases, $p \rightarrow q$ is T, and our expression also evaluates to T.
   \end{itemize}

Since we can express $\wedge$, $\vee$, and $\rightarrow$ using only $\leftrightarrow$ and $\neg$, and we already have $\neg$, the set $\{\leftrightarrow, \neg\}$ is adequate for propositional logic.



\newpage
\section*{With Lean 4}


\subsection*{Exercise 3 Write the script of the Lean 4 proof for Example 1.3.4 in Macbeth’s book. The book
gives the “proof by hand” in full, but does not give its mechanized version in Lean 4.}

\subsection*{Solutions in one file at: \url{https://github.com/nich-ikech/CS511-hw-macbeth/blob/d7a44938a14b51e12a670a41a9c27f0f70de6e46/cs511HwSolutions/hw01/hw01.lean}}
\newpage
\subsection*{Script link for Exercise 3}
\url{https://github.com/nich-ikech/CS511-hw-macbeth/blob/d7a44938a14b51e12a670a41a9c27f0f70de6e46/cs511HwSolutions/hw01/hw01_3.lean}

\newpage
\subsection*{Exercise 4 Write the script of the Lean 4 proof for Example 1.3.9 in Macbeth’s book. Again
here, the book gives the “proof by hand”, but does not give its mechanized version in Lean 4.}
\subsection*{Script link for Exercise 4}
\url{https://github.com/nich-ikech/CS511-hw-macbeth/blob/d7a44938a14b51e12a670a41a9c27f0f70de6e46/cs511HwSolutions/hw01/hw01_4.lean}

\newpage
\subsection*{Problem 2: Do Exercise 7, in Section 1.3.11, in Macbeth’s book.}
\subsection*{Script link for Problem 2 - Exercise 7}
\url{https://github.com/nich-ikech/CS511-hw-macbeth/blob/d7a44938a14b51e12a670a41a9c27f0f70de6e46/cs511HwSolutions/hw01/hw01_p2.lean}
\end{document}
