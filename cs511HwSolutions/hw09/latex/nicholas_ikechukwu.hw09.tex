\documentclass{article}
\usepackage[letterpaper]{geometry} % Set the paper size to US Letter
\usepackage{amsmath, amssymb, bussproofs, qtree, booktabs, array, lastpage, fancyhdr}
\usepackage{mdframed}
\usepackage{tcolorbox}
\usepackage{hyperref}
\usepackage{enumitem}



% \defeqine a custom-proof environment

\newenvironment{proof}
{\begin{mdframed}[linewidth=0.5pt]\begin{enumerate}[label=\arabic*.,leftmargin=*]}
{\end{enumerate}\end{mdframed}}

\hypersetup{
  colorlinks=true,
  linkcolor=blue,
  filecolor=magenta,      
  urlcolor=cyan,
  pdftitle={Your Title Here},
  pdfpagemode=FullScreen,
}
 

% Add the new command here
\newcommand{\defeq}{\stackrel{\text{def}}{=}}
\newcommand{\proves}{\vdash}


\pagestyle{fancy}
\fancyhf{}
\rhead{Page \thepage}
\lhead{Nicholas  Ikechukwu - U71641768 - Boston University}
\cfoot{}

\begin{document}


\begin{center}
    \Large\textbf{Solutions to CS511 Homework 09}
    
    \vspace{0.5cm}
    
    \large Nicholas Ikechukwu - U71641768
    
    \vspace{0.3cm}
    
    \large November 07, 2024
\end{center}



\section*{Exercise 1. Open EML.Chapter 6.pdf: Do Exercise 107 on page 64. }


\begin{mdframed}
    Exercise 107: (Two-Colorability of Graphs: First-Order Definable). The notion of two-colorable
    simple graphs coincides with the notion of bipartite simple graphs. Write an infinite set $\Gamma_{bipartite}$
    of first-order sentences such that, for every simple graph G, it holds that G $\models$ $\Gamma_{bipartite}$ iff G is
    bipartite.

    \vspace{1em} 
    Hint: G is bipartite iff every cycle in G (possibly with repeated vertices) has even length. 
\end{mdframed}
\section*{Solution:}
Let $\Gamma_{\text{bipartite}}$ be the set of first-order sentences that express that for every cycle of length $n$ (where $n$ is odd), such a cycle cannot exist in the graph. For each odd $n \geq 3$, we include a sentence $\phi_n$ in $\Gamma_{\text{bipartite}}$:
    
    \[
    \phi_n := \forall x_1 \ldots \forall x_n \left(\bigwedge_{i=1}^{n-1} E(x_i,x_{i+1}) \land E(x_n,x_1) \rightarrow \bigvee_{1 \leq i < j \leq n} x_i \approx x_j\right)
    \]
    
    Then:
    \[
    \Gamma_{\text{bipartite}} := \{\phi_n \mid n \geq 3 \text{ and } n \text{ is odd}\}
    \]
    
    This works because:
\begin{proof}
    
    \begin{itemize}
        \item Each $\phi_n$ says "there cannot be a cycle of length $n$" where $n$ is odd
        \item The formula enforces that if we have $n$ vertices connected in a cycle, at least two must be the same vertex
        \item A graph models $\Gamma_{\text{bipartite}}$ if and only if it has no odd cycles
        \item By the characterization of bipartite graphs, a graph is bipartite if and only if it has no odd cycles
    \end{itemize}
    
    Therefore, $G \models \Gamma_{\text{bipartite}}$ if and only if $G$ is bipartite.
\end{proof}


\newpage



\section*{Exercise 2. [LCS, page 163]: Do Exercise 2.4.6 (the last on that page).}

\begin{mdframed}
    Consider the three sentences:
    
    \vspace{1em}
    $\phi1 \defeq \forall x P (x, x)$
    
    \vspace{1em}
    $\phi2 \defeq \forall x \forall y (P (x, y) \rightarrow P (y, x))$
    
    \vspace{1em}
    $\phi3 \defeq \forall x \forall y \forall z ((P (x, y) \land P (y, z) \rightarrow P (x, z)))$
    
    \vspace{1em}

    which express that the binary predicate P is reflexive, symmetric and transitive,
    respectively. Show that none of these sentences is semantically entailed by the
    other ones by choosing for each pair of sentences above a model which satisfies
    these two, but not the third sentence – essentially, you are asked to find three
    binary relations, each satisfying just two of these properties. 
  
\end{mdframed}

\subsection*{Solution:}
We can show that none of these sentences semantically entails the others, by first finding three different models:
    
\vspace{1em}
1. A model satisfying $\phi_2$ and $\phi_3$ but not $\phi_1$ (symmetric and transitive but not reflexive)

2. A model satisfying $\phi_1$ and $\phi_3$ but not $\phi_2$ (reflexive and transitive but not symmetric)

3. A model satisfying $\phi_1$ and $\phi_2$ but not $\phi_3$ (reflexive and symmetric but not transitive)


\vspace{1em}
Now, we'll construct these models using simple binary relations on small sets:
    
\vspace{1em}

\textbf{Model 1:} (symmetric and transitive but not reflexive)
\begin{itemize}
    \item Domain: $A = \{1, 2\}$
    \item Relation: $P = \emptyset$ (empty relation)
    \item This is symmetric (vacuously) and transitive (vacuously) but not reflexive since $P(1,1)$ and $P(2,2)$ don't hold   ß
\end{itemize}

    
\textbf{Model 2:} (reflexive and transitive but not symmetric)

\begin{itemize}
\item Domain: $A = \{1, 2\}$
    \item Relation: $P = \{(1,1), (2,2), (1,2)\}$
    \item This is reflexive (all $(x,x)$ included) and transitive, but not symmetric since $(1,2)$ is in $P$ but $(2,1)$ is not
\end{itemize}

\textbf{Model 3:} (reflexive and symmetric but not transitive)
\begin{itemize}
    \item Domain: $A = \{1, 2, 3\}$
    \item Relation: $P = \{(1,1), (2,2), (3,3), (1,2), (2,1)\}$
    \item This is reflexive (all $(x,x)$ included) and symmetric, but not transitive since $(1,2)$ and $(2,3)$ are in $P$ but $(1,3)$ is not
\end{itemize}
    

Therefore, it is clear that each sentence is independent of the others.
\newpage

\section*{PROBLEM 1  Open Lecture Slides 26: Do the two parts of the exercise on page 7.}

\section*{Part 1:}
\begin{mdframed}
    Exercise:
    Let $\phi(x, y)$ be an atomic WFF with free variables x and y, and f a unary function symbol not appearing in $\phi$.
    \vspace{1em}

    1. Show that the sentence $\forall x \phi(x, f (x)) \rightarrow \forall x\exists y \phi(x, y)$ is semantically
    valid, i.e., the following sequent is formally derivable:
    \vspace{1em}

    $\vdash \forall x \phi(x, f (x)) \rightarrow \forall x \exists y \phi(x, y)$
    \vspace{1em}

    Hint: Use any of the available methods, i.e., try to find a formal proof
    or try a semantic approach to show $\models \forall x \phi(x, f (x)) \rightarrow \forall x\exists y \phi(x, y)$
    and then invoke the completeness of the proof rules.
\end{mdframed}

\section*{Solution:}
\textbf{1.} To show $\vdash \forall x \phi(x, f(x)) \rightarrow \forall x \exists y \phi(x, y)$


\vspace{1em}
is valid, we use a semantic approach:

\vspace{1em}
Suppose $M \models \forall x \phi(x, f(x))$ for some model $M$. 

\vspace{1em}
We need to show 
\[\begin{aligned}
    M \models \forall x \exists y \phi(x, y)
\end{aligned}\]

\vspace{1em}
Let $a$ be any element in the universe of $M$. 

We need to show $M \models \exists y \phi(a, y)$.

\vspace{1em}
From $M \models \forall x \phi(x, f(x))$, 

\vspace{1em}
we know that $M \models \phi(a, f(a))$.

\vspace{1em}
Let $b = f^M(a)$. Then $M \models \phi(a, b)$, which means $M \models \exists y \phi(a, y)$.

\vspace{1em}
Since $a$ was arbitrary, $M \models \forall x \exists y \phi(x, y)$.

\vspace{1em}
Therefore, by completeness, 
\[\begin{aligned}
    & \vdash \forall x \phi(x, f(x)) \rightarrow \forall x \exists y \phi(x, y)
\end{aligned}\]
\newpage


\section*{Part 2:}


\begin{mdframed}
    2. Show that the sentence $\forall x \exists y \phi(x, y) \rightarrow \forall x \phi(x, f (x))$ is NOT
    semanticalle valid, i.e., the following sequent is NOT derivable:
      \vspace{1em}
      
    $\vdash \forall x\exists y \phi(x, y) \rightarrow \forall x \phi(x, f (x))$
      \vspace{1em}
      
    Hint: Try a semantic approach, i.e., define an appropriate $\phi$ and a
    model where the left-hand side of “$\rightarrow$” is true but the right-hand side
    of “$\rightarrow$” is false, and then invoke the completeness of the proof rules.
\end{mdframed}
\section*{Solution:}
\textbf{2.} To show the sentence is not valid, we construct a counterexample:

\vspace{1em}
Let $\phi(x,y)$ be the atomic formula $x < y$ interpreted over the real numbers $\mathbb{R}$.

\vspace{1em}
Define $f(x) = x$ for all $x \in \mathbb{R}$.

\vspace{1em}
Then:
\begin{itemize}
\item $\forall x \exists y \phi(x,y)$ is true because for any real number $x$, there exists a larger real number $y$
\item However, $\forall x \phi(x,f(x))$ is false because $\phi(x,f(x))$ means $x < x$, which is false for all $x$
\end{itemize}

Therefore, we have found a model where $\forall x \exists y \phi(x,y)$ is true but $\forall x \phi(x,f(x))$ is false.

By completeness, 
\[\begin{aligned}
    & \not \vdash \forall x \exists y \phi(x,y) \rightarrow \forall x \phi(x,f(x))
\end{aligned}\]

\newpage




\section*{ON LEAN-4}
\subsection*{Solutions in one file at: 
\url{https://github.com/nich-ikech/CS511-hw-macbeth/blob/main/cs511HwSolutions/hw09/hw09_nicholas_ikechukwu.lean}}

\newpage

\section*{Exercise 3. From Macbeth’s book:}
\section*{Solutions}
\url{https://github.com/nich-ikech/CS511-hw-macbeth/blob/main/cs511HwSolutions/hw09/hw09_nicholas_ikechukwu.lean}

\newpage

\section*{Exercise 4. From Macbeth's book}

\url{https://github.com/nich-ikech/CS511-hw-macbeth/blob/main/cs511HwSolutions/hw09/hw09_nicholas_ikechukwu.lean}

\newpage

\section*{PROBLEM 2. From Macbeth's book}
\section*{Solution}

\url{https://github.com/nich-ikech/CS511-hw-macbeth/blob/main/cs511HwSolutions/hw09/hw09_nicholas_ikechukwu.lean}
\end{document}
