\documentclass{article}
\usepackage[letterpaper]{geometry} % Set the paper size to US Letter
\usepackage{amsmath, amssymb, bussproofs, qtree, booktabs, array, lastpage, fancyhdr}
\usepackage{mdframed}
\usepackage{tcolorbox}
\usepackage{hyperref}
\usepackage{enumitem}



% Define a custom-proof environment

\newenvironment{proof}
{\begin{mdframed}[linewidth=0.5pt]\begin{enumerate}[label=\arabic*.,leftmargin=*]}
{\end{enumerate}\end{mdframed}}

\hypersetup{
  colorlinks=true,
  linkcolor=blue,
  filecolor=magenta,      
  urlcolor=cyan,
  pdftitle={Your Title Here},
  pdfpagemode=FullScreen,
}
 

% Add the new command here
\newcommand{\defeq}{\stackrel{\text{def}}{=}}
\newcommand{\proves}{\vdash}


\pagestyle{fancy}
\fancyhf{}
\rhead{Page \thepage}
\lhead{Nicholas  Ikechukwu - U71641768 - Boston University}
\cfoot{}

\begin{document}


\begin{center}
    \Large\textbf{Solutions to CS511 Homework 08}
    
    \vspace{0.5cm}
    
    \large Nicholas Ikechukwu - U71641768
    
    \vspace{0.3cm}
    
    \large October 31, 2024
\end{center}



\section*{Exercise 1. [LCS, page 159-160]: Exercise 2.2.3. Do both part (a), on page 159, and part (b),
on page 160. }

\section*{Analysis of First-Order Formulas}

\section*{Part (a): Valid and Invalid Formulas in Predicate Logic}

Given: $m$ is a constant, $f$ is a function symbol with one argument, and $S$ and $B$ are two predicate symbols, each with two arguments.

\subsection*{Valid Formulas with Parse Trees}

\textbf{i.} $S(m, x)$ is a valid formula:

\vspace{1em}
\Tree [.$\bullet$
    [.$S$ ]
    [.$\circ$ $m$ ]
    [.$\circ$ $x$ ]
]

\textbf{ii.} $B(m, f(m))$ is a valid formula:

\vspace{1em}
\Tree [.$\bullet$
    [.$B$ ]
    [.$\circ$ $m$ ]
    [.$\bullet$
        [.$f$ ]
        [.$\circ$ $m$ ]
    ]
]

\textbf{vi.} $(B(x, y) \rightarrow (\exists z S(z,y)))$ is a valid formula:


\Tree [.$\bullet$
    [.$\rightarrow$
        [.$\bullet$
            [.$B$ ]
            [.$\circ$ $x$ ]
            [.$\circ$ $y$ ]
        ]
        [.$\bullet$
            [.$\exists z$ ]
            [.$\bullet$
                [.$S$ ]
                [.$\circ$ $z$ ]
                [.$\circ$ $y$ ]
            ]
        ]
    ]
]

\textbf{vii.} $(S(x, y) \rightarrow S(y, f(f(x))))$ is a valid formula:

\vspace{1em}
\Tree [.$\bullet$
    [.$\rightarrow$
        [.$\bullet$
            [.$S$ ]
            [.$\circ$ $x$ ]
            [.$\circ$ $y$ ]
        ]
        [.$\bullet$
            [.$S$ ]
            [.$\circ$ $y$ ]
            [.$\bullet$
                [.$f$ ]
                [.$\bullet$
                    [.$f$ ]
                    [.$\circ$ $x$ ]
                ]
            ]
        ]
    ]
]

\subsection*{Invalid Formulas with Reasons}

\textbf{iii.} $f(m)$ is not a formula because:
\begin{itemize}
    \item It is a term, not a formula
    \item No predicate symbol is applied
\end{itemize}

\textbf{iv.} $B(B(m, x), y)$ is not a formula because:
\begin{itemize}
    \item $B$ is a predicate symbol, not a function symbol
    \item Cannot use predicate $B$ as argument to another predicate
\end{itemize}

\textbf{v.} $S(B(m), z)$ is not a formula because:
\begin{itemize}
    \item $B(m)$ is invalid as $B$ requires two arguments
    \item Cannot use predicate as argument
\end{itemize}

\textbf{viii.} $(B(x) \rightarrow B(B(x)))$ is not a formula because:
\begin{itemize}
    \item $B$ requires two arguments but given only one
    \item Cannot use predicate $B$ as argument
\end{itemize}


\newpage


\section*{Part (b)}

Let $c$ and $d$ be constants, $f$ a function symbol with one argument, $g$ a function symbol with two arguments, $h$ a function symbol with three arguments, and $P$ and $Q$ are predicate symbols with three arguments.

\subsection*{Valid Formulas with Parse Trees}

\textbf{i.} $\forall x P(f(d), h(g(c, x), d, y))$ is a valid formula:

\vspace{1em}

\Tree [.$\bullet$
    [.$\forall x$ ]
    [.$P$
        [.$\bullet$
            [.$f$
                [.$\circ$ $d$ ]
            ]
        ]
        [.$\bullet$
            [.$h$
                [.$\bullet$
                    [.$g$
                        [.$\circ$ $c$ ]
                        [.$\circ$ $x$ ]
                    ]
                ]
                [.$\circ$ $d$ ]
                [.$\circ$ $y$ ]
            ]
        ]
    ]
]

\vspace{1em}

\textbf{iii.} $\forall x Q(g(h(x, f(d), x), g(x, x)), h(x, x, x), c)$ is a valid formula:

\Tree [.$\bullet$
    [.$\forall x$ ]
    [.$Q$
        [.$\bullet$
            [.$g$
                [.$\bullet$
                    [.$h$
                        [.$\circ$ $x$ ]
                        [.$\bullet$
                            [.$f$
                                [.$\circ$ $d$ ]
                            ]
                        ]
                        [.$\circ$ $x$ ]
                    ]
                ]
                [.$\bullet$
                    [.$g$
                        [.$\circ$ $x$ ]
                        [.$\circ$ $x$ ]
                    ]
                ]
            ]
        ]
        [.$\bullet$
            [.$h$
                [.$\circ$ $x$ ]
                [.$\circ$ $x$ ]
                [.$\circ$ $x$ ]
            ]
        ]
        [.$\circ$ $c$ ]
    ]
]

\vspace{1em}

\textbf{vi.} $Q(c, d, c)$ is a valid formula:

\vspace{1em}

\Tree [.$\bullet$
    [.$Q$ ]
    [.$\circ$ $c$ ]
    [.$\circ$ $d$ ]
    [.$\circ$ $c$ ]
]

\subsection*{Invalid Formulas with Reasons}

\textbf{ii.} $\forall x P(f(d), h(P(x, y), d, y))$ is not a formula because:
\begin{itemize}
    \item Cannot use predicate $P$ as argument to function $h$
    \item Predicates can only appear as atomic formulas
\end{itemize}

\textbf{iv.} $\exists z (Q(z, z, z) \rightarrow P(z))$ is not a formula because:
\begin{itemize}
    \item $P$ requires three arguments but given only one
    \item Arity mismatch for predicate $P$
\end{itemize}

\textbf{v.} $\forall x \forall y (g(x, y) \rightarrow P(x, y, x))$ is not a formula because:
\begin{itemize}
    \item $g(x, y)$ is a term, not a formula
    \item Cannot use $\rightarrow$ with a term on left side
    \item Only atomic formulas can be connected by logical operators
\end{itemize}

\newpage
\section*{Exercise 2.}

\subsection*{Hint: Review the hint in the preceding exercise}



\begin{mdframed}
\end{mdframed}

\section*{Part 1: [LCS, page 160]: Exercise 2.3.2. }


\vspace{1em}


\vspace{1em}

\newpage

\section*{Part 2: [LCS, page 160]: Exercise 2.3.3, modified as follows. Change part (c) to read “at least three
distinct elements”.}
\subsection*{[LCS, page 160]: Exercise 2.3.2.}




\section*{2.3.2. Formula Interpretation}

The formula $\exists x \exists y (\neg(x = y) \land (\forall z ((z = x) \lor (z = y))))$ specifies:

\vspace{1em}
"There exist x and y, such that there is no x that equals y and for all z, z either equals x or y".

\vspace{1em}
in essense:

\vspace{1em}
"There exist exactly two distinct elements in the model."

\vspace{1em}
This is because:
\begin{itemize}
    \item $\exists x \exists y (\neg(x = y))$ states there are at least two different elements
    \item $\forall z ((z = x) \lor (z = y))$ states every element must be equal to either $x$ or $y$
    \item Together, they specify that there are exactly two distinct elements
\end{itemize}

\newpage
\subsection*{[LCS, page 160]: Exercise 2.3.3, modified as follows. Change part (c) to read “at least three
distinct elements”}
\section*{2.3.3. Predicate Logic Sentences}

\subsection*{(a) Exactly three distinct elements:}
\[\begin{aligned}
& \exists x \exists y \exists z (\neg(x = y) \land \neg(y = z) \land \neg(x = z) \land \\
& \forall w ((w = x) \lor (w = y) \lor (w = z)))
\end{aligned}\]

\subsection*{(b) At most three distinct elements:}
\[\forall x \forall y \forall w \forall z ((w = x) \lor (w = y) \lor (w = z))\]

\subsection*{(c) At least three distinct elements:}
\[\exists x \exists y \exists z (\neg(x = y) \land \neg(y = z) \land \neg(x = z))\]

\textbf{Explanations:}
\begin{itemize}
    \item For "exactly three": We state there exist three distinct elements AND every element must be one of these three
    \item For "at most three": We state that any fourth element must be equal to one of three elements
    \item For "at least three": We state there exist three elements that are all different from each other
\end{itemize}

\vspace{1em}


\vspace{1em}

\newpage

\section*{PROBLEM 1 Open EML.Chapter 6.pdf. Do part Exercise 99 on page 61.}

\section*{Interpretation of PL in FOL, II (Solution)}

Let $\Sigma' \defeq \{f, g_1, g_2, g_3, c_1, c_2\}$ where:
\begin{itemize}
    \item $f$ corresponds to $\neg$ (unary)
    \item $g_1$ corresponds to $\land$ (binary)
    \item $g_2$ corresponds to $\lor$ (binary)
    \item $g_3$ corresponds to $\to$ (binary)
    \item $c_1$ corresponds to $\bot$ (constant)
    \item $c_2$ corresponds to $\top$ (constant)
\end{itemize}

To define the two-element structure up to isomorphism, we construct $\psi$ as follows:

\[\begin{aligned}
\psi \defeq & \hspace{1em} \exists x \exists y (\neg(x \approx y) \land \forall z(z \approx x \lor z \approx y)) \land \\
& \forall x \forall y ( \\
& \quad (x \approx c_1 \lor x \approx c_2) \land \\
& \quad \neg(c_1 \approx c_2) \land \\
& \quad (f(c_1) \approx c_2 \land f(c_2) \approx c_1) \land \\
& \quad (g_1(c_1, c_1) \approx c_1 \land g_1(c_1, c_2) \approx c_1 \land \\
& \quad g_1(c_2, c_1) \approx c_1 \land g_1(c_2, c_2) \approx c_2) \land \\
& \quad (g_2(c_1, c_1) \approx c_1 \land g_2(c_1, c_2) \approx c_2 \land \\
& \quad g_2(c_2, c_1) \approx c_2 \land g_2(c_2, c_2) \approx c_2) \land \\
& \quad (g_3(c_1, c_1) \approx c_2 \land g_3(c_1, c_2) \approx c_2 \land \\
& \quad g_3(c_2, c_1) \approx c_1 \land g_3(c_2, c_2) \approx c_2))
\end{aligned}\]

For any propositional formula $\phi$, we construct $\phi'$ as:
\[\phi' \defeq \phi \land \psi\]

\textbf{Proof of Correctness:}
\begin{itemize}
    \item The sentence $\psi$ ensures:
    \begin{itemize}
        \item Exactly two distinct elements ($c_1$ and $c_2$ representing false and true)
        \item The truth tables for negation ($f$), conjunction ($g_1$), disjunction ($g_2$), and implication ($g_3$)
        \item The constants false ($c_1$) and true ($c_2$)
    \end{itemize}
    \item If $\phi$ is valid in PL, then it evaluates to true under all truth assignments
    \item The sentence $\psi$ ensures that any model of $\phi'$ is isomorphic to the two-element Boolean algebra
    \item Therefore, $\phi$ is valid in PL if and only if $\phi'$ is valid in FOL
\end{itemize}

\newpage







\section*{ON LEAN-4}
\subsection*{Solutions in one file at: 
\url{https://github.com/nich-ikech/CS511-hw-macbeth/blob/main/cs511HwSolutions/hw08/hw08_nicholas_ikechukwu.lean}}

\newpage

\section*{Exercise 3. Two closely related parts, which you have to do in Lean:}
\section*{Solutions}
\url{https://github.com/nich-ikech/CS511-hw-macbeth/blob/main/cs511HwSolutions/hw08/hw08_nicholas_ikechukwu.lean}

\newpage

\section*{Exercise 4. From Macbeth's book}

\url{https://github.com/nich-ikech/CS511-hw-macbeth/blob/main/cs511HwSolutions/hw08/hw08_nicholas_ikechukwu.lean}

\newpage

\section*{PROBLEM 2. From Macbeth's book}
\section*{Solution}

\url{https://github.com/nich-ikech/CS511-hw-macbeth/blob/main/cs511HwSolutions/hw08/hw08_nicholas_ikechukwu.lean}
\end{document}
