\documentclass{article}
\usepackage[letterpaper]{geometry} % Set the paper size to US Letter
\usepackage{amsmath, amssymb, bussproofs, qtree, booktabs, array, lastpage, fancyhdr}
\usepackage{mdframed}
\usepackage{tcolorbox}
\usepackage{hyperref}
\usepackage{enumitem}



% Define a custom-proof environment

\newenvironment{proof}
{\begin{mdframed}[linewidth=0.5pt]\begin{enumerate}[label=\arabic*.,leftmargin=*]}
{\end{enumerate}\end{mdframed}}

\hypersetup{
  colorlinks=true,
  linkcolor=blue,
  filecolor=magenta,      
  urlcolor=cyan,
  pdftitle={Your Title Here},
  pdfpagemode=FullScreen,
}

\newcommand{\proves}{\vdash}


\pagestyle{fancy}
\fancyhf{}
\rhead{Page \thepage}
\lhead{Nicholas  Ikechukwu - U71641768}
\cfoot{}

\begin{document}


\begin{center}
    \Large\textbf{Solutions to CS511 Homework 05}
    
    \vspace{0.5cm}
    
    \large Nicholas Ikechukwu - U71641768
    
    \vspace{0.3cm}
    
    \large October 09, 2024
\end{center}



\section*{Exercise 1. [LCS, page 160]: Exercise 2.3.1, do parts (a) and (b) only }
\begin{mdframed}
    Prove the validity of the following sequents using, among others, the rules =i
    and =e. Make sure that you indicate for each application of =e what the rule
    instances $\phi$, $t_1$ and $t_2$ are.
    
    Use $\approx$, instead of =, for the formal symbol whose interpretation is equality. In LaTeX, you can typeset
    with ${"approx"}$
     
    
    \begin{enumerate}[label=(\alph*)]
    \item $(y = 0) \land (y = x) \vdash 0 = x$
    \item $t_1 = t_2 \vdash (t + t_2) = (t + t_1)$
    \end{enumerate}
\end{mdframed}

\section*{Solutions:}

\section*{(a) $(y \approx 0) \wedge (y \approx x) \vdash 0 \approx x$}

\begin{proof}
\begin{enumerate}
    \item $(y \approx 0) \wedge (y \approx x)$ \hfill [Premise]
    \item $y \approx 0$ \hfill [$\wedge$ elimination, 1]
    \item $y \approx x$ \hfill [$\wedge$ elimination, 1]
    \item $0 \approx y$ \hfill [=e: $\varphi(z) := (z \approx y)$, $t_1 := y$, $t_2 := 0$, from 2]
    \item $0 \approx x$ \hfill [=e: $\varphi(z) := (0 \approx z)$, $t_1 := y$, $t_2 := x$, from 4, 3]
\end{enumerate}
\end{proof}

\newpage


\section*{(b) $t_1 \approx t_2 \vdash (t + t_2) \approx (t + t_1)$}

\begin{proof}
\begin{enumerate}
    \item $t_1 \approx t_2$ \hfill [Premise]
    \item $(t + t_1) \approx (t + t_1)$ \hfill [=i]
    \item $(t + t_2) \approx (t + t_1)$ \hfill [=e: $\varphi(z) := ((t + z) \approx (t + t_1))$, $t_1 := t_1$, $t_2 := t_2$, from 1, 2]
\end{enumerate}
\end{proof}



\newpage

\section*{Exercise 2. LCS, page 161: Exercise 2.3.9, do parts (a) and (d) only.}
\begin{mdframed}
    Prove the validity of the following sequents in predicate logic, where F , G, P ,
    and Q have arity 1, and S has arity 0 (a ‘propositional atom’):

    \begin{itemize}
        \item (a) $\exists x (S \rightarrow Q(x)) \vdash S \rightarrow \exists x Q(x)$
        \item (d) $\forall x P(x) \rightarrow S \vdash \exists x (P(x) \rightarrow S)$
    \end{itemize}

\end{mdframed}

\section*{Solutions:}


\section*{(a) $\exists x(S \to Q(x)) \vdash S \to \exists xQ(x)$}

Let $\mathcal{I}$ be any interpretation in which $\exists x(S \to Q(x))$ is true.

\begin{proof}
    \item There exists some element $a$ in the domain such that $S \to Q(a)$ is true in $\mathcal{I}$. \hfill
    \item To prove $S \to \exists xQ(x)$, consider two cases for $S$: \hfill
    \item Case 1: If $S$ is false in $\mathcal{I}$, then $S \to \exists xQ(x)$ is trivially true. \hfill
    \item Case 2: If $S$ is true in $\mathcal{I}$, then: \hfill
    \begin{enumerate}
        \item $Q(a)$ must be true in $\mathcal{I}$ (from steps 1 and 4). \hfill
        \item Therefore, $\exists xQ(x)$ is true in $\mathcal{I}$. \hfill
        \item Hence, $S \to \exists xQ(x)$ is true in $\mathcal{I}$. \hfill
    \end{enumerate}
    \item In both cases, $S \to \exists xQ(x)$ is true in $\mathcal{I}$. \hfill
Thus, whenever $\exists x(S \to Q(x))$ is true in an interpretation, $S \to \exists xQ(x)$ is also true in that interpretation, proving the validity of the sequent. \hfill
\end{proof}

\newpage
\section*{(d) $\forall xP(x) \to S \vdash \exists x(P(x) \to S)$}
    We prove this by contradiction:

\begin{proof}
    \item Assume there exists an interpretation $\mathcal{I}$ in which $\forall xP(x) \to S$ is true but $\exists x(P(x) \to S)$ is false. \hfill
    \item In $\mathcal{I}$, $\forall x\neg(P(x) \to S)$ must be true (negation of $\exists x(P(x) \to S)$). \hfill
    \item This means for every element $a$ in the domain of $\mathcal{I}$: \hfill
    \begin{enumerate}
        \item $P(a)$ is true and $S$ is false. \hfill
    \end{enumerate}
    \item Therefore, $\forall xP(x)$ is true in $\mathcal{I}$. \hfill
    \item From steps 1 and 4, $S$ must be true in $\mathcal{I}$ (by modus ponens). \hfill
    \item But this contradicts step 3(a), where $S$ is false. \hfill
This contradiction shows that our assumption in step 1 must be false. Therefore, in any interpretation where $\forall xP(x) \to S$ is true, $\exists x(P(x) \to S)$ must also be true, proving the validity of the sequent. \hfill
\end{proof}

\newpage
\section*{PROBLEM 1: Let $\psi1, \psi2, and \psi3$ be the three axioms of group theory, which are written as
first-order wff’s on page 11 of Lecture Slides 20. Let $\psi$ be the wff in the middle of the same page
11 of Lecture Slides 20. The wff $\psi$ expresses the uniqueness of inverses in groups. Your task is to
produce a formal proof, as a natural deduction, of the following judgment:
$\psi1, \psi2, \psi3 \vdash \psi$.}

\subsection*{Hint: Do Exercises 1 and 2 above before this problem. Also use $\approx$ for the formal symbol whose
interpretation is equality, leaving = for equality at the meta-level.
}


\section*{Solution:}
Let $\psi_1$, $\psi_2$, and $\psi_3$ be the three axioms of group theory:

\begin{enumerate}
    \item $\forall x (e \cdot x \approx x \wedge x \cdot e \approx x)$ (identity)
    \item $\forall x \exists y (x \cdot y \approx e \wedge y \cdot x \approx e)$ (inverse)
    \item $\forall x \forall y \forall z ((x \cdot y) \cdot z \approx x \cdot (y \cdot z))$ (associative)
\end{enumerate}

Let $\phi$ be the wff expressing the uniqueness of inverses in groups:

\[\phi \equiv \forall x \forall y \forall z (x \cdot y \approx e \wedge x \cdot z \approx e \to y \approx z)\]

We need to prove: $\psi_1, \psi_2, \psi_3 \proves \phi$
Let $x$, $y$, and $z$ be arbitrary elements of the group.

\begin{proof}

    \item Assume $x \cdot y \approx e$ and $x \cdot z \approx e$.
    \item From $\psi_2$, there exists an element $x'$ such that $x \cdot x' \approx e$ and $x' \cdot x \approx e$.
    \item By substituting $e$ into the equation, we have $(x \cdot y) \cdot x' \approx e$.
    \item By the identity axiom ($\psi_1$), $(x \cdot y) \cdot x' = x'$.
    \item By the associative property ($\psi_3$), we have $x \cdot (y \cdot x') = x'$.
    \item Therefore, since $e = y$, we can conclude that $y = x'$.
    \item Similarly, we can show that $z = x'$.
    \item Thus, we conclude that $y = z$.

\end{proof}
Since $x$, $y$, and $z$ were arbitrary, we have proven $\phi$.


\newpage







\section*{ON LEAN-4}
\subsection*{Solutions in one file at: 
\url{https://github.com/nich-ikech/CS511-hw-macbeth/blob/main/cs511HwSolutions/hw05/hw05_nicholas_ikechukwu.lean}}

\newpage

\section*{Exercise 3. Hint: These should be easy if you read the book. Use existential quantifiers.}
\section*{Solution}
\url{https://github.com/nich-ikech/CS511-hw-macbeth/blob/main/cs511HwSolutions/hw05/hw05_nicholas_ikechukwu.lean}

\newpage

\section*{Exercise 4. Hint: These use existential and universal quantifiers. The existential quantifiers are used in both
context and goal, but universal quantifiers only in context.}

\url{https://github.com/nich-ikech/CS511-hw-macbeth/blob/main/cs511HwSolutions/hw05/hw05_nicholas_ikechukwu.lean}

\newpage

\section*{PROBLEM 2. Prove in Lean 4 the judgment for which you produced a formal proof as a natural
deduction in Problem 1 above.}
\section*{Solution}

\url{https://github.com/nich-ikech/CS511-hw-macbeth/blob/main/cs511HwSolutions/hw05/hw05_nicholas_ikechukwu.lean}
\end{document}
