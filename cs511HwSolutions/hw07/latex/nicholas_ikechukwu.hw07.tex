\documentclass{article}
\usepackage[letterpaper]{geometry} % Set the paper size to US Letter
\usepackage{amsmath, amssymb, bussproofs, qtree, booktabs, array, lastpage, fancyhdr}
\usepackage{mdframed}
\usepackage{tcolorbox}
\usepackage{hyperref}
\usepackage{enumitem}



% Define a custom-proof environment

\newenvironment{proof}
{\begin{mdframed}[linewidth=0.5pt]\begin{enumerate}[label=\arabic*.,leftmargin=*]}
{\end{enumerate}\end{mdframed}}

\hypersetup{
  colorlinks=true,
  linkcolor=blue,
  filecolor=magenta,      
  urlcolor=cyan,
  pdftitle={Your Title Here},
  pdfpagemode=FullScreen,
}

% Add the new command here
\newcommand{\defeq}{\stackrel{\text{def}}{=}}
\newcommand{\proves}{\vdash}


\pagestyle{fancy}
\fancyhf{}
\rhead{Page \thepage}
\lhead{Nicholas  Ikechukwu - U71641768 - Boston University}
\cfoot{}

\begin{document}


\begin{center}
    \Large\textbf{Solutions to CS511 Homework 07}
    
    \vspace{0.5cm}
    
    \large Nicholas Ikechukwu - U71641768
    
    \vspace{0.3cm}
    
    \large October 24, 2024
\end{center}



\section*{Exercise 1. Open Lecture Slides 29, I, Analytical Tableaux for Classical First-Order
Logic. Do Exercise 1 on page 14. }

\subsection*{Hint: To show that $\{\phi1, \phi2, \phi3\} \models \phi$ is equivalent to showing $\{\phi1, \phi2, \phi3\}  \vdash \phi $ (by completeness),
which is equivalent to showing $\vdash (\phi1 \land \phi2 \land \phi3) \rightarrow \phi.$ These equivalences hold for formal proofs carried
out according to the rules of natural deduction, and they hold again when analytic tableaux are used
as a formal-proof system.}

\begin{mdframed}
    Show that $\Gamma \models \phi$ 
    \vspace{1em}

    where:
    \vspace{1em}

    $\Gamma = \{\forall x\forall y\forall z(P(x,y) \land P(y,z) \to P(x,z)), \forall x\forall y(P(x,y) \to P(y,x))\}$
    
    $\phi = \forall x\forall y\forall z(P(x,y) \land P(z,y) \to P(x,z))$
\end{mdframed}

\section*{Solutions to First-Order Ground Tableaux Proof:}

\vspace{1em}

We negate the formula we want to prove and show it leads to a contradiction:

\vspace{1em}
\Tree [.{1. $\neg(\forall x\forall y\forall z(P(x,y) \land P(z,y) \to P(x,z)))$} 
    [.{2. $\exists x\exists y\exists z(P(x,y) \land P(z,y) \land \neg P(x,z))$}
        [.{3. $P(x_1,y_1) \land P(z_1,y_1) \land \neg P(x_1,z_1)$}
            [.{4. $P(x_1,y_1)$}
                [.{5. $P(z_1,y_1)$}
                    [.{6. $\neg P(x_1,z_1)$}
                        [.{7. $P(y_1,z_1)$ \quad (from 2, 5)}
                            [.{8. $P(x_1,z_1)$ \quad (from 1, 4, 7)}
                                [.{X} ]
                            ]
                        ]
                    ]
                ]
            ]
        ]
    ]
]

\vspace{1em}


Explanation:
\begin{enumerate}
    \item Line 1: Negation of the conclusion
    \item Line 2: Converting universal quantifier to existential
    \item Line 3: Instantiation with fresh variables $x_1$, $y_1$, $z_1$
    \item Lines 4-6: Conjunction elimination
    \item Line 7: From $P(z_1,y_1)$ and symmetry axiom
    \item Line 8: From $P(x_1,y_1)$, $P(y_1,z_1)$ and transitivity axiom
    \item X: Contradiction between lines 6 and 8
\end{enumerate}

Since we reached a contradiction, the original formula is valid.


\newpage

\section*{Exercise 2. Open Lecture Slides 29, I, Analytical Tableaux for Classical First-Order
Logic. Do Exercise 2 on page 14.}

\subsection*{Hint: Review the hint in the preceding exercise}



\begin{mdframed}
    Show that $\Gamma \models \phi$ 
    \vspace{1em}

    where:
    \vspace{1em}

    $\Gamma = \{\forall x Q(a,x,x), \forall x\forall y\forall z(Q(x,y,z) \to Q(x,s(y),s(z))), \forall x\forall y\forall z(Q(x,y,z) \to Q(y,x,z))\}$
    
    $\phi = \exists x Q(s^{(2)}(a),s^{(3)}(a),x)$
    \vspace{1em}

    where $Q$ is a ternary predicate symbol, $s$ is a unary function symbol, and $a$ is a constant symbol.
\end{mdframed}

\section*{Solution using ground tableaux method: First-Order Ground Tableaux Proof}


\vspace{1em}


\Tree [.{1. $\neg(\exists x Q(s^{(2)}(a),s^{(3)}(a),x))$}
    [.{2. $\forall x Q(a,x,x)$}
        [.{3. $Q(a,a,a)$ \quad [from 2]}
            [.{4. $Q(a,s(a),s(a))$ \quad [from 2,3]}
                [.{5. $Q(s(a),a,s(a))$ \quad [from 3,4]}
                    [.{6. $Q(s(a),s^{(2)}(a),s^{(2)}(a))$ \quad [from 4,5]}
                        [.{7. $Q(s^{(2)}(a),s(a),s^{(2)}(a))$ \quad [from 5,6]}
                            [.{8. $Q(s^{(2)}(a),s^{(2)}(a),s^{(3)}(a))$ \quad [from 6,7]}
                                [.{9. $Q(s^{(2)}(a),s^{(3)}(a),s^{(3)}(a))$ \quad [from 7,8]}
                                    [.{10. $\forall x \neg Q(s^{(2)}(a),s^{(3)}(a),x)$ \quad [from 1]}
                                        [.{11. $\neg Q(s^{(2)}(a),s^{(3)}(a),s^{(3)}(a))$ \quad [from 10]}
                                            [.{X} ]
                                        ]
                                    ]
                                ]
                            ]
                        ]
                    ]
                ]
            ]
        ]
    ]
]

\vspace{1em}
The tableau closes because we derived a contradiction between lines 9 and 11, proving that $\Gamma \models \phi$.

\newpage


\section*{PROBLEM 1 Open EML.Chapter 6.pdf. Do part Exercise 113 on page 69}

\subsection*{Hint: This is a continuation of the discussion in lecture yesterday (Thursday, October 17). As sug-
gested in lecture, read carefully and understand Example 112 on pages 68-69 before embarking on
Exercise 113.
}



\section*{Queens Problem II}

\subsection*{Summary}
We need to construct a first-order sentence $\psi$ that characterizes solutions of the Queens Problem using a unary function $q$ instead of a binary relation. The structure $M = (N, =, +, <, 0, q)$ interprets $q$ as a function where $q(i) = 0$ means no queen in row $i$, and $q(i) = j$ means a queen is placed at position $(i,j)$.

\subsection*{Solution}
The first-order sentence $\psi$ can be defined as:

\[ \psi \defeq \psi_{\text{base}} \wedge (\psi_{\text{fin}} \vee \psi_{\text{inf}}) \wedge \psi_{\text{valid}} \wedge \psi_{\text{noattack}} \]

where:

\begin{align*}
\psi_{\text{base}} &\defeq q(0) \approx 0 \\
\\
\psi_{\text{fin}} &\defeq \exists n > 0.(\forall i.(i > n \rightarrow q(i) \approx 0) \wedge \\
    &\qquad \forall i.(0 < i \leq n \rightarrow 0 < q(i) \leq n)) \\
\\
\psi_{\text{inf}} &\defeq \forall i > 0.\exists j > 0.(q(i) \approx j) \\
\\
\psi_{\text{valid}} &\defeq \forall i > 0.\forall j > 0.(q(i) \approx j \rightarrow \\
    &\qquad \forall k > 0.(k \not\approx i \rightarrow q(k) \not\approx j)) \\
\\
\psi_{\text{noattack}} &\defeq \forall i_1 > 0.\forall i_2 > 0.(i_1 \not\approx i_2 \wedge q(i_1) \not\approx 0 \wedge q(i_2) \not\approx 0 \rightarrow \\
    &\qquad |i_1 - i_2| \not\approx |q(i_1) - q(i_2)|)
\end{align*}

where:
\begin{itemize}
\item $\psi_{\text{base}}$ ensures $q(0) = 0$
\item $\psi_{\text{fin}}$ handles finite solutions
\item $\psi_{\text{inf}}$ handles infinite solutions
\item $\psi_{\text{valid}}$ ensures no two queens share a column
\item $\psi_{\text{noattack}}$ ensures no two queens are on the same diagonal
\end{itemize}

The absolute value notation $|x - y|$ can be expressed in first-order logic as:
\[ |x - y| \approx z \defeq ((x - y \approx z) \vee (y - x \approx z)) \wedge z \geq 0 \]


\newpage







\section*{ON LEAN-4}
\subsection*{Solutions in one file at: 
\url{https://github.com/nich-ikech/CS511-hw-macbeth/blob/main/cs511HwSolutions/hw07/hw07_nicholas_ikechukwu.lean}}

\newpage

\section*{Exercise 3. From Macbeth's book}
\section*{Solution}
\url{https://github.com/nich-ikech/CS511-hw-macbeth/blob/main/cs511HwSolutions/hw07/hw07_nicholas_ikechukwu.lean}

\newpage

\section*{Exercise 4. From Macbeth's book}

\url{https://github.com/nich-ikech/CS511-hw-macbeth/blob/main/cs511HwSolutions/hw07/hw07_nicholas_ikechukwu.lean}

\newpage

\section*{PROBLEM 2. From Macbeth's book}
\section*{Solution}

\url{https://github.com/nich-ikech/CS511-hw-macbeth/blob/main/cs511HwSolutions/hw07/hw07_nicholas_ikechukwu.lean}
\end{document}
