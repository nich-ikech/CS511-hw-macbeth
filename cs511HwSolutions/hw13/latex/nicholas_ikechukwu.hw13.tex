\documentclass{article}
\usepackage[letterpaper]{geometry} % Set the paper size to US Letter
\usepackage{amsmath, amssymb, bussproofs, qtree, booktabs, array, lastpage, fancyhdr}
\usepackage{mdframed}
\usepackage{tcolorbox}
\usepackage{hyperref}
\usepackage{enumitem}
\usepackage{stmaryrd}
\usepackage{graphicx}




% \defeqine a custom-proof environment

\newenvironment{proof}
{\begin{mdframed}[linewidth=0.5pt]\begin{enumerate}[label=\arabic*.,leftmargin=*]}
{\end{enumerate}\end{mdframed}}

\hypersetup{
  colorlinks=true,
  linkcolor=blue,
  filecolor=magenta,      
  urlcolor=cyan,
  pdftitle={Your Title Here},
  pdfpagemode=FullScreen,
}
 

% Add the new command here
\newcommand{\defeq}{\stackrel{\text{def}}{=}}
\newcommand{\proves}{\vdash}


\pagestyle{fancy}
\fancyhf{}
\rhead{Page \thepage}
\lhead{Nicholas  Ikechukwu - U71641768 - Boston University}
\cfoot{}

\setlength{\parindent}{0pt}

\begin{document}


\begin{center}
    \Large\textbf{Solutions to CS511 Homework 13}
    
    \vspace{0.5cm}
    
    \large Nicholas Ikechukwu - U71641768
    
    \vspace{0.3cm}
    
    \large December 12, 2024
\end{center}



\section*{Exercise 1. Some questions related to database methods}

\begin{mdframed}
    \vspace{1em}
        \textbf{Exercise:} Part (a) The SQL standard provides an operation EXISTS, which can be used as an existential
        quantifier. For example,
        SELECT ... FROM ... WHERE EXISTS \( < subquery > \)
        So to express a database, we certainly need at least first-order logic. Argue as to whether or not
        second -order logic or higher is needed for any SQL operations you are familiar with. Is first-order
        logic sufficient for all SQL operations?
    \vspace{1em}
\end{mdframed}


\section*{Part (a). Solution}

\newpage

\begin{mdframed}
    \vspace{1em}
        \textbf{Exercise:} Part (b) Considering the following schema:
        Draw a diagram representing this schema, using the diagram language from the lecture on December
        3. Use filled circles for table vertices and empty circles for type vertices  
\end{mdframed}
\begin{figure}[h!]
    \centering
    \includegraphics[width=0.5\textwidth]{1b_image.png} % Replace 'image-name' with your image file name (without extension)
    \label{fig:sample-image}
\end{figure}  

\section*{Part (b). Solution}

\newpage


\section*{Exercise 2. }

\begin{mdframed}In 1763, Leonhard Euler created a very famous graph representing the islands of the
    Pregel River and the seven bridges across it. (To understand the very simple question he wanted to
    solve which motivated one of the first problems answered by graph theory, you may read the Wikipedia
    article on “The Seven Bridges of K¨onigsberg.”) Here is the graph K which he drew, with some arrows
    added:
\end{mdframed}
   
\begin{figure}[h!]
    \centering
    \includegraphics[width=0.3\textwidth]{2_image.png} % Replace 'image-name' with your image file name (without extension)
    \label{fig:sample-image}
\end{figure} 

Let $\kappa$ be the free category on K.
\section*{Part (a):} List the 15 morphisms of K along with their domains and codomains.
\subsection*{Part (a). Solution:}


\newpage

\section*{Part (b):} A diagram is called a “commutative” diagram if all paths with the same start and end point
are equal; that is, if all “parallel” paths through the diagram produce the same result. Let $\kappa$' be the
commutative free category on K; list the morphisms of $\kappa$'. Using the answers of part (a) should make
this a trivial exercise.

\subsection*{Part (b). Solution:}



\newpage

\section*{Part (c):} Consider the following category V:
U - p $\rightarrow$ V $\leftarrow$ q - W \\
Define a functor F : V → $\kappa$.


\subsection*{Part (c). Solution:}


\newpage

\section*{Part (d):} Imagine a category that looks like this:
\begin{figure}[h!]
    \centering
    \includegraphics[width=0.3\textwidth]{2d_image.png} % Replace 'image-name' with your image file name (without extension)
    \label{fig:sample-image}
\end{figure} 

Argue why there cannot be a functor from this category to $\kappa$.

\subsection*{Part (d). Solution:}






\newpage

\section*{PROBLEM 1. Adjunction Between Functors  }


\begin{mdframed}
    \textbf{Exercise }:
    
    \vspace{1em}
The preceding example about currying illustrates of what is called an \textbf{adjunction} between functors, here the functor $- \times B$ and the functor $(-)^B$. We only said how each of these two functors works on objects: For an arbitrary set $X$, the first functor returns the set $X \times B$ while the second returns the set $X^B$.

There are three parts in this problem -- these may look scary to you, but the answer to each part takes at most 2 (or perhaps 3) lines:

\begin{enumerate}
    \item Given a morphism $f : X \to Y$, what morphism should $- \times B : X \times B \to Y \times B$ return?
    
    \vspace{1em}
    \item Given a morphism $f : X \to Y$, what morphism should $(-)^B : X^B \to Y^B$ return?
    
    \vspace{1em}
    \item Consider the function $+ : \mathbb{N} \times \mathbb{N} \to \mathbb{N}$, which maps $(a,b)$ to $a + b$. Currying $+$ we get a function $p : \mathbb{N} \to \mathbb{N}^{\mathbb{N}}$. What is $p(3)$?
\end{enumerate}
\end{mdframed}
    

 \subsection*{Part 1:  } 
 Given a morphism $f : X \to Y$, what morphism should $- \times B : X \times B \to Y \times B$ return? 
 \subsection*{Part (1). Solution:}


 \newpage

 \subsection*{Part 2:  } 
 Given a morphism $f : X \to Y$, what morphism should $(-)^B : X^B \to Y^B$ return?
  \subsection*{Part (2). Solution:}


 \newpage


 \subsection*{Part 3:  } 
 Consider the function $+ : \mathbb{N} \times \mathbb{N} \to \mathbb{N}$, which maps $(a,b)$ to $a + b$. Currying $+$ we get a function $p : \mathbb{N} \to \mathbb{N}^{\mathbb{N}}$. What is $p(3)$?
  \subsection*{Part (3). Solution:}


 



\newpage



\section*{ON LEAN-4}
\subsection*{Solutions in one file at: 
\url{https://github.com/nich-ikech/CS511-hw-macbeth/blob/main/cs511HwSolutions/hw13/hw13_nicholas_ikechukwu.lean}}
 
\newpage
\section*{Exercise 3. From Macbeth’s book: Exercise 10.1.5.4 }
\section*{Solutions}
\url{https://github.com/nich-ikech/CS511-hw-macbeth/blob/main/cs511HwSolutions/hw13/hw13_nicholas_ikechukwu.lean}



\newpage

\section*{Exercise 4. From Macbeth’s book: Exercise 10.1.5.5 }
\section*{Solutions}
\url{https://github.com/nich-ikech/CS511-hw-macbeth/blob/main/cs511HwSolutions/hw13/hw13_nicholas_ikechukwu.lean}

\newpage


\section*{PROBLEM 2. From Macbeth’s book: Exercise 10.1.5.6}
\section*{Solutions}

\url{https://github.com/nich-ikech/CS511-hw-macbeth/blob/main/cs511HwSolutions/hw13/hw13_nicholas_ikechukwu.lean}


\end{document}
