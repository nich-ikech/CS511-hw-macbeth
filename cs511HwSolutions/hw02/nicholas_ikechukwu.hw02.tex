\documentclass{article}
\usepackage[letterpaper]{geometry} % Set the paper size to US Letter
\usepackage{amsmath, amssymb, bussproofs, qtree, booktabs, array, lastpage, fancyhdr}
\usepackage{mdframed}
\usepackage{tcolorbox}
\usepackage{hyperref}
\usepackage{enumitem}



% Define a custom-proof environment

\newenvironment{proof}
{\begin{mdframed}[linewidth=0.5pt]\begin{enumerate}[label=\arabic*.,leftmargin=*]}
{\end{enumerate}\end{mdframed}}

\hypersetup{
  colorlinks=true,
  linkcolor=blue,
  filecolor=magenta,      
  urlcolor=cyan,
  pdftitle={Your Title Here},
  pdfpagemode=FullScreen,
}

\newcommand{\proves}{\vdash}

\pagestyle{fancy}
\fancyhf{}
\rhead{Page \thepage}
\lhead{Nicholas  Ikechukwu - U71641768}
\cfoot{}

\begin{document}


\begin{center}
    \Large\textbf{Solutions to CS511 Homework 01}
    
    \vspace{0.5cm}
    
    \large Nicholas Ikechukwu - U71641768
    
    \vspace{0.3cm}
    
    \large September 19, 2024
\end{center}



\section*{Exercise 1 Go to page 9 in Lecture Slides 06. Your task is to carefully write all the details of
the proof by structural induction. These details are not included in the slides.}

\section*{Solution}
\section*{Concise Proof by Structural Induction}

\textbf{Proposition:} $\forall s, t \in A^*, \text{reverse}(s \cdot t) = \text{reverse}(t) \cdot \text{reverse}(s)$

Let $P(t) := \forall s \in A^*, \text{reverse}(s \cdot t) = \text{reverse}(t) \cdot \text{reverse}(s)$

\textbf{Proof by structural induction on $t$:}

\begin{enumerate}
    \item Base case: $t = \varepsilon$
    
    $\forall s \in A^*, \text{reverse}(s \cdot \varepsilon) = \text{reverse}(s) = \varepsilon \cdot \text{reverse}(s) = \text{reverse}(\varepsilon) \cdot \text{reverse}(s)$

    \item Inductive step: Assume $P(t)$ holds for $t \in A^*$. Show $P(a \cdot t)$ for $a \in A$.
    
    $\forall s \in A^*$:
    \begin{align*}
    \text{reverse}(s \cdot (a \cdot t)) 
    &= \text{reverse}((s \cdot a) \cdot t) &&\text{[associativity]} \\
    &= \text{reverse}(t) \cdot \text{reverse}(s \cdot a) &&\text{[I.H.]} \\
    &= \text{reverse}(t) \cdot (\text{reverse}(a) \cdot \text{reverse}(s)) &&\text{[def. of reverse]} \\
    &= (\text{reverse}(t) \cdot \text{reverse}(a)) \cdot \text{reverse}(s) &&\text{[associativity]} \\
    &= \text{reverse}(a \cdot t) \cdot \text{reverse}(s) &&\text{[def. of reverse]}
    \end{align*}
\end{enumerate}

By structural induction, $P(t)$ holds $\forall t \in A^*$, proving the proposition.

\newpage

\section*{Exercise 2 [LCS, page 87]: Exercise 1.4.15.
Hint: You may find it helpful to review pages 20 and 21 in Lecture Slides 02.}
\section*{Solution}

\section*{Concise Proof by Mathematical Induction}
\begin{mdframed}
\textbf{Theorem:} For $n \geq 1$, 
\[((\varphi_1 \land (\varphi_2 \land (\cdots\land \varphi_n)\cdots) \to \psi) \to (\varphi_1 \to (\varphi_2 \to (\cdots(\varphi_n \to \psi)\cdots))))\]
\end{mdframed}

Let $P(n)$ denote the theorem statement.

\textbf{Proof:}

\begin{enumerate}
    \item Base case $(n = 1)$:
    \begin{mdframed}
    $P(1): ((\varphi_1 \to \psi) \to (\varphi_1 \to \psi))$ [Trivially true]
    \end{mdframed}

    \item Inductive step:
    Assume $P(k)$ holds for some $k \geq 1$.
    To prove $P(k+1)$:

    LHS of $P(k+1)$:
    \begin{mdframed}
    \begin{align*}
    &(\varphi_1 \land (\varphi_2 \land (\cdots\land \varphi_{k+1})\cdots) \to \psi) \\
    &\equiv ((\varphi_1 \land (\varphi_2 \land (\cdots\land \varphi_k)\cdots)) \land \varphi_{k+1} \to \psi) \\
    &\equiv (\varphi_1 \land (\varphi_2 \land (\cdots\land \varphi_k)\cdots) \to (\varphi_{k+1} \to \psi)) \quad \text{[Deduction theorem]}
    \end{align*}
    \end{mdframed}

    Applying $P(k)$ to this:
    \begin{mdframed}
    $(\varphi_1 \to (\varphi_2 \to (\cdots(\varphi_k \to (\varphi_{k+1} \to \psi))\cdots)))$
    \end{mdframed}

    This is the RHS of $P(k+1)$.
\end{enumerate}

Therefore, by mathematical induction, $P(n)$ holds for all $n \geq 1$.


\newpage
\section*{PROBLEM 1 Show that any of the three rules 
{(LEM),(PBC),(¬¬E)} are interderivable.}
\section*{Solution}

\section*{Interderivability of LEM, PBC, and ¬¬E}

We will show that the three rules Law of Excluded Middle (LEM), Proof by Contradiction (PBC), and Double Negation Elimination (¬¬E) are interderivable.

\subsection*{(a) (PBC) is derivable from (¬¬E)}

\begin{proof}
\item $\neg\varphi \to \bot$ \hfill given
\item $\neg\varphi$ \hfill assumption
\item $\bot$ \hfill $\to$E 1, 2
\item $\neg\neg\varphi$ \hfill $\neg$I 2-3
\item $\varphi$ \hfill ¬¬E 4
\end{proof}

\subsection*{(b) (LEM) is derivable from (PBC)}

\begin{proof}
\item $\neg(\varphi \lor \neg\varphi)$ \hfill assumption
\item \quad $\varphi$ \hfill assumption
\item \quad $\varphi \lor \neg\varphi$ \hfill $\lor$I 2
\item \quad $\bot$ \hfill $\neg$E 1, 3
\item $\neg\varphi$ \hfill $\neg$I 2-4
\item $\varphi \lor \neg\varphi$ \hfill $\lor$I 5
\item $\bot$ \hfill $\neg$E 1, 6
\item $\varphi \lor \neg\varphi$ \hfill PBC 1-7
\end{proof}

\subsection*{(c) (¬¬E) is derivable from (LEM)}

\begin{proof}
\item $\neg\neg\varphi$ \hfill premise
\item $\varphi \lor \neg\varphi$ \hfill LEM
\item \quad $\varphi$ \hfill assumption
\item \quad $\varphi$ \hfill reiteration 3
\item \quad $\neg\varphi$ \hfill assumption
\item \quad $\bot$ \hfill $\neg$E 1, 5
\item \quad $\varphi$ \hfill $\bot$E 6
\item $\varphi$ \hfill $\lor$E 2, 3-4, 5-7
\end{proof}

Therefore, we have shown that (¬¬E) $\Rightarrow$ (PBC) $\Rightarrow$ (LEM) $\Rightarrow$ (¬¬E), proving that these three rules are interderivable.




\newpage
\subsection*{Solutions in one file at: 
\url{https://github.com/nich-ikech/CS511-hw-macbeth/blob/d7a44938a14b51e12a670a41a9c27f0f70de6e46/cs511HwSolutions/hw01/hw01.lean}}

\newpage
\section*{Exercise 3 For each of the three examples in the following three sections of Macbeth’s book, your
task is to remove ‘sorry’ and insert appropriate Lean 4 tactics}
\section*{Solution}


\newpage
\section*{Exercise 4 For each of the three examples in the following three sections of Macbeth’s book, your
task is to remove ‘sorry’ and insert appropriate Lean 4 tactics.}
\section*{Solution}

\newpage
\section*{PROBLEM 2 For each of the three examples in the following three sections of Macbeth’s book,
your task is to remove ‘sorry’ and insert appropriate Lean 4 tactics}
\section*{Solution}

\end{document}
