\documentclass{article}
\usepackage[letterpaper]{geometry} % Set the paper size to US Letter
\usepackage{amsmath, amssymb, bussproofs, qtree, booktabs, array, lastpage, fancyhdr}
\usepackage{mdframed}
\usepackage{tcolorbox}
\usepackage{hyperref}
\usepackage{enumitem}
\usepackage{stmaryrd}



% \defeqine a custom-proof environment

\newenvironment{proof}
{\begin{mdframed}[linewidth=0.5pt]\begin{enumerate}[label=\arabic*.,leftmargin=*]}
{\end{enumerate}\end{mdframed}}

\hypersetup{
  colorlinks=true,
  linkcolor=blue,
  filecolor=magenta,      
  urlcolor=cyan,
  pdftitle={Your Title Here},
  pdfpagemode=FullScreen,
}
 

% Add the new command here
\newcommand{\defeq}{\stackrel{\text{def}}{=}}
\newcommand{\proves}{\vdash}


\pagestyle{fancy}
\fancyhf{}
\rhead{Page \thepage}
\lhead{Nicholas  Ikechukwu - U71641768 - Boston University}
\cfoot{}

\setlength{\parindent}{0pt}

\begin{document}


\begin{center}
    \Large\textbf{Solutions to CS511 Homework 12}
    
    \vspace{0.5cm}
    
    \large Nicholas Ikechukwu - U71641768
    
    \vspace{0.3cm}
    
    \large December 05, 2024
\end{center}



\section*{Exercise 1. Do the exercise on page 42.}

\begin{mdframed}
    \vspace{1em}
        \textbf{Exercise:} Define \( X \sim Y \) differently in second-order logic by asserting the existence of a unary function \( F \) from \( X \) to \( Y \) which is both injective and surjective.
    \vspace{1em}
\end{mdframed}


\section*{Solution to: Defining $X \sim Y$ in Second-Order Logic}

If we want to define $X \sim Y$ using a bijective function $F: X \to Y$, we can express it as follows:

\[
X \sim Y \equiv \exists F. (\forall x_1, x_2 \in X. (F(x_1) = F(x_2) \rightarrow x_1 = x_2)) \land (\forall y \in Y. \exists x \in X. F(x) = y)
\]

\vspace{1em}
Let's' break the formula down into two separate parts:

\vspace{1em}
1. \textbf{Injectivity}: Here, the condition $\forall x_1, x_2 \in X. (F(x_1) = F(x_2) \rightarrow x_1 = x_2)$ ensures that $F$ is injective, meaning each element in $X$ maps to a unique element in $Y$.

\vspace{1em}
2. \textbf{Surjectivity}: For this, the condition $\forall y \in Y. \exists x \in X. F(x) = y$ ensures that $F$ is surjective, meaning every element in $Y$ has a pre-image in $X$.

\vspace{1em}
The conditions, together, ensure that $F$ is a bijection from $X$ to $Y$, thereby establishing the equivalence relation $X \sim Y$.

\newpage


\section*{Exercise 2. both parts of the exercise on page 47.}

\begin{mdframed}
    \vspace{1em}
        \textbf{Exercise }: 
        
\vspace{1em}
        1. Define a second-order sentence \( \Psi_{\text{countable-infty}} \) such that \( A \models \Psi_{\text{countable-infty}} \) if and only if \( A \) is countably infinite.

\vspace{1em}
        2. Define a second-order sentence \( \Psi_{\text{uncountable}} \) such that \( A \models \Psi_{\text{uncountable}} \) if and only if \( A \) is uncountably infinite.

\vspace{1em}
        Note that \( \Psi_{\text{countable-infty}} \) and \( \Psi_{\text{uncountable}} \) in this exercise are sentences, i.e., closed well-formed formulas (wff's) which do not contain any free variables.

    \vspace{1em}
\end{mdframed}
    


\section*{Solution:}

\textbf{Part 1. Countably Infinite Set }:

\vspace{1em}
If we want to define \( \Psi_{\text{countable-infty}} \), basically, we need to express that a set \( A \) is countably infinite. 
We do this by simply ensuring that \( A \) is infinite and every infinite subset of \( A \) has a bijection to \( A \). The second-order sentence is:

\vspace{1em}
\[
\Psi_{\text{countable-infty}} = \Phi_{\text{infty}}(A) \land (\forall X \subseteq A. (\Phi_{\text{infty}}(X) \rightarrow (X \sim A)))
\]

\vspace{1em}
Where:
- \( \Phi_{\text{infty}}(A) \) asserts that \( A \) is infinite.
- \( X \sim A \) shows there exists a bijection between \( X \) and \( A \).

\vspace{1em}
\newpage


\textbf{Part 2. Uncountably Infinite Set }:

\vspace{1em}
If we want to define \( \Psi_{\text{uncountable}} \), we also want to express that a set \( A \) is uncountably infinite by negating the condition for countability:

\vspace{1em}
\[
\Psi_{\text{uncountable}} = \Phi_{\text{infty}}(A) \land (\neg (\forall X \subseteq A. (\Phi_{\text{infty}}(X) \rightarrow (X \sim A))))
\]

\vspace{1em}
Where:
- The negation of the second part ensures that there exists at least one infinite subset of \( A \) that does not have a bijection with \( A \), indicating uncountability.

\vspace{1em}

\newpage


\section*{Exercise 3. Do part 1 only of the exercise on page 50.}

\begin{mdframed}
    \vspace{1em}
        \textbf{Exercise }: Write a second-order wff \( \theta(x, y) \) such that:

        \[
        \theta(x, y) \iff \text{"no binary predicate } Y \text{ can discern } x \text{ and } y\text{"}
        \]
        
        Your task here is to write a wff of second-order logic modeling the English phrase to the right of "iff".
        
    \vspace{1em}
\end{mdframed}
    


\section*{Solution:}

For us to express that no binary predicate \( Y \) can discern \( x \) and \( y \), we basically need to ensure that 
for any binary relation \( Y \), the truth value of \( Y(x, z) \) matches \( Y(y, z) \) for all possible \( z \), 
and similarly, \( Y(z, x) \) matches \( Y(z, y) \). We can express this as:

\vspace{1em}
\[
\theta(x, y) = \forall Y. (\forall z. (Y(x, z) \leftrightarrow Y(y, z)) \land (Y(z, x) \leftrightarrow Y(z, y)))
\]

\vspace{1em}
In the formula above:
- This first part, \( \forall z. (Y(x, z) \leftrightarrow Y(y, z)) \), ensures that for any element \( z \), 
if \( x \) is related to \( z \), then \( y \) must also be related to \( z \), and vice versa.

\vspace{1em}
- The second part, \( (Y(z, x) \leftrightarrow Y(z, y)) \), inturn,  ensures symmetry in the sense that if any element \( z \) 
is related to \( x \), it must also be related to \( y \), and vice versa.

\vspace{1em}
Clearly, the above ensures that no binary predicate can distinguish between the elements \( x \) and \( y \).


\newpage

\section*{PROBLEM 1. Open [LCS, page 165]: Do parts (a), (b), and (c) only in Exercise 2.6.5.\newline\\
Hint: Part (c) is already done in Lecture Slides 32, pp 20-24. You may want to do it differently! }


\begin{mdframed}
    \textbf{Exercise }: 
    
    \vspace{1em}
    Let \( P \) and \( R \) be predicate symbols of arity 2. Write formulas of existential second-order logic of the form \( \exists P \, \psi \) that hold in all models of the form \( M = (A, R^M) \) if and only if:

    \vspace{1em}
    (a) \( R \) contains a reflexive and symmetric relation.

    \vspace{1em}
    (b) \( R \) contains an equivalence relation.

    \vspace{1em}
    (c) There is an \( R \)-path that visits each node of the graph exactly once – such a path is called Hamiltonian. 
(Using a different approach from Lecture Slides 32, pp 20-24).
\end{mdframed}
    
\section*{Solution:}

 \subsection*{Part a:  Reflexive and Symmetric Relation} 
 For us to express that \( R \) contains a reflexive and symmetric relation, we define:

\[
\exists P. (\forall x. P(x, x)) \land (\forall x \forall y. (P(x, y) \rightarrow P(y, x)))
\]

Our formula asserts:

\vspace{1em}
- Reflexivity: \( \forall x. P(x, x) \)

    \vspace{1em}
- Symmetry: \( \forall x \forall y. (P(x, y) \rightarrow P(y, x)) \)

    \newpage

 \subsection*{Part b: Equivalence Relation} 
 An equivalence relation is reflexive, symmetric, and transitive. Therefore:

\[
\exists P. (\forall x. P(x, x)) \land (\forall x \forall y. (P(x, y) \rightarrow P(y, x))) \land (\forall x \forall y \forall z. ((P(x, y) \land P(y, z)) \rightarrow P(x, z)))
\]

This formula includes:

    \vspace{1em}
- Reflexivity: \( \forall x. P(x, x) \)

    \vspace{1em}
- Symmetry: \( \forall x \forall y. (P(x, y) \rightarrow P(y, x)) \)

    \vspace{1em}
- Transitivity: \( \forall x \forall y \forall z. ((P(x, y) \land P(y, z)) \rightarrow P(x, z)) \)

 \newpage

 \subsection*{Part c.} 
 We can express the existence of a Hamiltonian path using a different approach thus:

    \vspace{1em}
\[
\exists P. \left( \forall x \exists! y. P(x, y) \right) \land \left( \forall x \exists! z. P(z, x) \right) \land \left( \forall x \forall y. (P(x, y) \rightarrow R(x, y)) \right)
\]
 
    \vspace{1em}
 Where:
 
    \vspace{1em}
- \( P(x, y) \) is a binary predicate indicating that \( y \) is the successor of \( x \) in the path. 

    \vspace{1em}
- \textbf{Unique Successor}: For every node \( x \), there exists exactly one node \( y \) such that \( P(x, y) \).

    \vspace{1em}
 - \textbf{Unique Predecessor}: For every node \( x \), there exists exactly one node \( y \) such that \( R(x,y)\).
 
 \vspace{1em}
 - \textbf{Path Condition}: For every pair of nodes \( x, y \), if \( P(x, y) \), then there must be an edge between them in \( R(x, y) \).
 \vspace{1em}

\newpage



\section*{ON LEAN-4}
\subsection*{Solutions in one file at: 
\url{https://github.com/nich-ikech/CS511-hw-macbeth/blob/main/cs511HwSolutions/hw12/hw12_nicholas_ikechukwu.lean}}
 
\newpage

\section*{Exercise 4. From Macbeth’s book:}
\section*{Solutions}
\url{https://github.com/nich-ikech/CS511-hw-macbeth/blob/main/cs511HwSolutions/hw12/hw12_nicholas_ikechukwu.lean}

\newpage

\section*{Exercise 5. From Macbeth's book}
\section*{Solutions}
\url{https://github.com/nich-ikech/CS511-hw-macbeth/blob/main/cs511HwSolutions/hw12/hw12_nicholas_ikechukwu.lean}

\newpage


\section*{Exercise 6. From Macbeth’s book:}
\section*{Solutions}
\url{https://github.com/nich-ikech/CS511-hw-macbeth/blob/main/cs511HwSolutions/hw12/hw12_nicholas_ikechukwu.lean}

\newpage

\section*{PROBLEM 2. From Macbeth's book}
\section*{Solutions}

\url{https://github.com/nich-ikech/CS511-hw-macbeth/blob/main/cs511HwSolutions/hw12/hw12_nicholas_ikechukwu.lean}


\end{document}
