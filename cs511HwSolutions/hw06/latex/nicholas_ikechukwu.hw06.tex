\documentclass{article}
\usepackage[letterpaper]{geometry} % Set the paper size to US Letter
\usepackage{amsmath, amssymb, bussproofs, qtree, booktabs, array, lastpage, fancyhdr}
\usepackage{mdframed}
\usepackage{tcolorbox}
\usepackage{hyperref}
\usepackage{enumitem}



% Define a custom-proof environment

\newenvironment{proof}
{\begin{mdframed}[linewidth=0.5pt]\begin{enumerate}[label=\arabic*.,leftmargin=*]}
{\end{enumerate}\end{mdframed}}

\hypersetup{
  colorlinks=true,
  linkcolor=blue,
  filecolor=magenta,      
  urlcolor=cyan,
  pdftitle={Your Title Here},
  pdfpagemode=FullScreen,
}

% Add the new command here
\newcommand{\defeq}{\stackrel{\text{def}}{=}}
\newcommand{\proves}{\vdash}


\pagestyle{fancy}
\fancyhf{}
\rhead{Page \thepage}
\lhead{Nicholas  Ikechukwu - U71641768 - Boston University}
\cfoot{}

\begin{document}


\begin{center}
    \Large\textbf{Solutions to CS511 Homework 06}
    
    \vspace{0.5cm}
    
    \large Nicholas Ikechukwu - U71641768
    
    \vspace{0.3cm}
    
    \large October 16, 2024
\end{center}



\section*{Exercise 1. Lecture Slides 22, FO Definability, Appendix. Ex. 3 on pg2. }

\subsection*{Hint: Carefully use the hint, as well as the special case given in lecture on Thursday, October 10.
Use $\approx$, instead of =, for the formal symbol whose interpretation is equality. In LaTeX, you can typeset
$\approx$ with \ approx.}

\begin{mdframed}
    Every finite subset X $\subseteq$ N is first-order definable in the structure ($\mathbb{N}; <$).
\end{mdframed}

\section*{Solutions to Ex.3 : First-Order Definability}
Ex.3. Every finite subset $X \subseteq \mathbb{N}$ is first-order definable in the structure $\mathbb{N}; <$. We can use the first-order definable constant 0 and the successor function.

% Define the WFF for the subset X
Let $\phi_{\{0\}}(x)$ define the constant 0 as:
$$
\phi_{\{0\}}(x) \defeq \forall y (x \approx y \lor x < y)
$$


% Define the successor function
Let $\phi_{\text{succ}}(x, y)$ define the successor function:
$$
\phi_{\text{succ}}(x, y) \defeq (x < y) \land \forall z (x < z \rightarrow (y \approx z \lor y < z)) \land \forall z (z < y \rightarrow (z \approx x \lor z < x))
$$


% Define the WFF for the finite subset
A finite subset $X = \{ a_1, a_2, \ldots, a_n \}$ can be defined as follows:
$$
\phi_X(x) \defeq \bigvee_{i=1}^{n} \phi_{\text{succ}}^{i-1}(0, x)
$$

where $\phi_{\text{succ}}^{i-1}(0, x)$ denotes applying the successor function $(i-1)$ times starting from 0.




\newpage

\section*{Exercise 2. Lecture Slides 22, FO Definability, Appendix. Ex 10 on page 3.}

\subsection*{Hint 1 : Review how we showed, in lecture on Thursday October 10, that the unary predicate prime
was first-order definable in in structure ($\mathbb{N}$; =, <, +, ·, S, 0). See pages 15-16 in Lecture Slides 21
where prime is called pr}

\subsection*{Hint 2 : You case use any of the preceding exercises in Lecture Slides 22, FO Definability,
Appendix as “lemmas” without proving them again, i.e., use the statements of the exercises as
premises that you do not need to prove.}

\begin{mdframed}
    Show that the predicate $\text{prime} : \mathbb{N} \to \{\text{true}, \text{false}\}$ is first-order definable in the structure $(\mathbb{N}; |, +, 0)$.
\end{mdframed}

\section*{Solution to Ex. 10}

\begin{proof}
    \item We define the predicate $\text{prime}(x)$ using the following first-order formula:
    
    \[\phi_{\text{prime}}(x) \defeq \neg(x \approx 1) \wedge \forall y \forall z [(y|x \wedge z|x) \to (y \approx 1 \vee z \approx 1 \vee y \approx x \vee z \approx x)]\]
    
    \item This formula uses only the divisibility relation $|$ and equality $\approx$. We need to show that the other components can be defined using $\{|, +, 0\}$:
    
    \item $x \approx 1$ can be defined as $\phi_1(x) \defeq \forall y (y|x)$
    
    \item $x \approx 0$ can be defined as $\phi_0(x) \defeq \forall y (x|y)$
    
    \item Using these, we can rewrite $\phi_{\text{prime}}(x)$ as:
    
    \[\phi_{\text{prime}}(x) \defeq \neg\phi_1(x) \wedge \forall y \forall z [(y|x \wedge z|x) \to (\phi_1(y) \vee \phi_1(z) \vee y \approx x \vee z \approx x)]\]
    
  
\end{proof}



\newpage
\section*{PROBLEM 1 Open EML.Chapter 4.pdf. Do part 2 and part 3 in Exercise 71 on page 46.}

\subsection*{Hint: You will have to do a fair amount of reading in the same chapter before doing this problem, in
particular, you should carefully study the material in Example 70 right before Exercise 71. However,
by the time you have finished your reading, your answers will be relatively easy to write.
}


\section*{Solution to Ex. 2:}
We can prove that if every finite subgraph of $G$ is $k$-colorable, then $G$ is $k$-colorable, using the Compactness Theorem for ZOL.
\begin{proof}
    
    
    \item Let $\Gamma = \Theta_1 \cup \Theta_2 \cup \Theta_3 \cup \Phi_1 \cup \Phi_2 \cup \Phi_3$ as defined in the background.
    
    \item Consider any finite subset $\Gamma_0 \subseteq \Gamma$. This $\Gamma_0$ only mentions a finite number of vertices of $G$.
    
    \item Let $G_0$ be the finite subgraph of $G$ induced by these vertices.
    
    \item By our assumption, $G_0$ is $k$-colorable.
    
    \item This $k$-coloring of $G_0$ can be extended to a model $M_0$ of $\Gamma_0$.
    
    \item Therefore, every finite subset of $\Gamma$ is satisfiable.
    
    \item By the Compactness Theorem for ZOL (Theorem 65), $\Gamma$ itself is satisfiable.
    
    \item Let $M$ be a model of $\Gamma$. By the Basic Herbrand Theorem (Theorem 58), we can assume $M$ is a Herbrand model.
    
    \item The reduct $M_0 = (M, R^M)$ contains an isomorphic copy of $G$ (by the Claim in the background).
    
    \item The interpretation of $C_1, \ldots, C_k$ in $M$ gives a valid $k$-coloring of this copy of $G$.
    
    \item Therefore, $G$ is $k$-colorable.
    \end{proof}

\newpage

\section*{Solution to Ex. 3:}
    Let $G = (N, E)$ be a planar graph. We will prove that $G$ is 4-colorable.

\begin{proof}
    
    \item Every finite subgraph of $G$ is planar (as planarity is a hereditary property).
    
    \item By the Four Color Theorem, every finite planar graph is 4-colorable.
    
    \item Therefore, every finite subgraph of $G$ is 4-colorable.
    
    \item By the result from Exercise 2, since every finite subgraph of $G$ is 4-colorable, $G$ itself is 4-colorable.
    
    
    \end{proof}

\newpage







\section*{ON LEAN-4}
\subsection*{Solutions in one file at: 
\url{https://github.com/nich-ikech/CS511-hw-macbeth/blob/main/cs511HwSolutions/hw06/hw06_nicholas_ikechukwu.lean}}

\newpage

\section*{Exercise 3. From Macbeth's book}
\section*{Solution}
\url{https://github.com/nich-ikech/CS511-hw-macbeth/blob/main/cs511HwSolutions/hw06/hw06_nicholas_ikechukwu.lean}

\newpage

\section*{Exercise 4. From Macbeth's book}

\url{https://github.com/nich-ikech/CS511-hw-macbeth/blob/main/cs511HwSolutions/hw06/hw06_nicholas_ikechukwu.lean}

\newpage

\section*{PROBLEM 2. From Macbeth's book}
\section*{Solution}

\url{https://github.com/nich-ikech/CS511-hw-macbeth/blob/main/cs511HwSolutions/hw06/hw06_nicholas_ikechukwu.lean}
\end{document}
